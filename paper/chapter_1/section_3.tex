% 1.3
% Notation and Conventions

Unless otherwise noted these are the standard conventions used in this thesis.
I use standard particle theory units of $\hbar=c=1$.
The four vector is given by Greek indices
\begin{equation}
  x^\mu=\left(\begin{matrix} x_0 \\ x_1 \\ x_2 \\ x_3\end{matrix}\right),
\end{equation}
where $x_0$ is the temporal component $t$ and the remaining three components are the spatial components of $\vec x$.
For the metric I use the west coast convention for the metric
\begin{equation}
  \eta_{\mu\nu} = \left(\begin{matrix}1& 0 & 0 & 0 \\
                                      0&-1 & 0 & 0 \\
                                      0& 0 &-1 & 0 \\
                                      0& 0 & 0 &-1\end{matrix}\right).
\end{equation}
Standard Einstein summation notation applies to contract indices.

I work in the chiral representation for the Dirac spin matrices $\gamma^\mu$
\begin{equation}
  \gamma^0=\left(\begin{matrix}0 & I \\ I & 0\end{matrix}\right),\quad\gamma^i=\left(\begin{matrix}0 & \sigma^i \\ -\sigma^i & 0 \end{matrix}\right),
\end{equation}
where $I$ is the $2\times 2$ identity matrix and $\sigma^i$ are the Pauli matrices:
\begin{equation}
  I=\left(\begin{matrix}1 & 0 \\ 0 & 1\end{matrix}\right),\quad\sigma^1=\left(\begin{matrix}0 & 1 \\ 1 & 0\end{matrix}\right),\quad\sigma^2=\left(\begin{matrix}0 & -i \\ i & 0\end{matrix}\right),\quad\sigma^3=\left(\begin{matrix}1 & 0 \\ 0 & -1\end{matrix}\right).
\end{equation}

I use the standard `slash' notation to contract four vectors with $\gamma^\mu$:
\begin{equation}
  \slashed{p} \equiv p_\mu\gamma^\mu.
\end{equation}
The momentum operator I use the notation
\begin{equation}
  p^\mu = i\partial^\mu,
\end{equation}
where
\begin{equation}
 \partial^\mu = \left(\frac{\partial}{\partial x^0}, -\nabla\right)=\left(\frac{\partial}{\partial x^0},\frac{\partial}{\partial x^1},\frac{\partial}{\partial x^2},\frac{\partial}{\partial x^3}\right).
\end{equation}

The Gell-Mann matricies $\lambda_a$ are:
\begin{equation}
  \begin{aligned}
    &\lambda_1 = \left(\begin{matrix} 0 & 1 & 0 \\ 1 & 0 & 0 \\ 0 & 0 & 0 \end{matrix}\right) &\lambda_2 = \left(\begin{matrix} 0 & -i & 0 \\ i & 0 & 0 \\ 0 & 0 & 0 \end{matrix}\right) \\\\
    &\lambda_3 = \left(\begin{matrix} 1 & 0 & 0 \\ 0 & -1 & 0 \\ 0 & 0 & 0 \end{matrix}\right) &\lambda_4 = \left(\begin{matrix} 0 & 0 & 1 \\ 0 & 0 & 0 \\ 1 & 0 & 0 \end{matrix}\right) \\\\
    &\lambda_5 = \left(\begin{matrix} 0 & 0 & -i \\ 0 & 0 & 0 \\ i & 0 & 0 \end{matrix}\right) &\lambda_6 = \left(\begin{matrix} 0 & 0 & 0 \\ 0 & 0 & 1 \\ 0 & 1 & 0 \end{matrix}\right) \\\\
    &\lambda_7 = \left(\begin{matrix} 0 & 0 & 0 \\ 0 & 0 & -i \\ 0 & i & 0 \end{matrix}\right) &\lambda_8 = \frac{1}{\sqrt{3}} \left(\begin{matrix} 1 & 0 & 0 \\ 0 & 1 & 0 \\ 0 & 0 & -2 \end{matrix}\right)
  \end{aligned}
\end{equation}

% 5.3
% Monte Carlo Renormaliztion Group

\subsection{Blocking}
In a lattice simulation the quantities stored in the computer are the link variables.
The link variable $U_{n,\mu}$ encodes the gauge field at sight $n$ in all four $\hat{\mu}$ directions.
A MCRG blocking step is a transformation that takes two links $U_{n,\mu}$ and $U_{n+\hat{\mu},\mu}$ and creates one link with twice the lattice spacing of the initial two links.
\begin{figure}[h]
  \centering
  \includegraphics[height=3in]{\cnum{4}/fig/block2.png}
  \caption{}
  \label{fig:block2}
\end{figure}
The blocked lattice will have a factor of $2^N$ less links than the original lattice.
In four dimensions this means that an observable such as a Wilson loop will have a factor of 16 less statistics on the blocked lattice compared to the original lattice.
Fortunately, on an $N$ dimensional hyper cubic lattice there are $2^N$ unique ways to perform this blocking.
Therefore in four dimensions there are 16 unique blocking configurations.
We can preserve statistics in our calculation by performing all of the unique blockings and averaging all of the observalbes from those blockings together.
Additionally we store the blocked links on a lattice of the original cardinality using offsets as shown in \ref{fig:block2}.

The blocked links have a lattice spacking of twice the original lattice, $a' = 2a$ such that the physical size of the box has not been changed.





% 3.2
% Gauge Fields

\subsection{Pure Gauge}
The simplest quantum field theory to put on the lattice is a pure Yang-Mills gauge theory.
In group theory this these are $SU(N_C)$ gauge theories without any pesky fermions.
The most straight forward discretization of a pure gauge action is in terms of the plaquette
\begin{equation}
  P_{x,\mu\nu}=\mbox{Tr}\Big[U_{x,\mu}U_{x+\hat{\mu},\nu}U^{\dagger}_{x+\hat{\nu},\mu}U^{\dagger}_{x,\nu}\Big],
\end{equation}
where $\mu\neq\nu$ and the trace over color and spin indices is implied.
The plaquette is the smallest closed loop and therefore the smallest gauge invariant observable, corresponding to the links multiplied around a unit square of the lattice.
The Hermitian conjugate, $P^{\dagger}_{x,\mu\nu}$, is the same loop with the order of the link products in the opposite direction.

To illustrate how the action is built out of plaquettes lets consider an abelian gauge theory for simplicity.
For an abelian gauge theory
\begin{align}
  F_{\mu\nu} &= \partial_\mu A_\nu-\partial_\nu A_\mu \\
  \Lagr &= \frac{1}{4}F_{\mu\nu}F^{\mu\nu} \\
  S &= \int \Lagr d^4x = \int  \frac{1}{4}F_{\mu\nu}F^{\mu\nu}
\end{align}
Introducing the lattice spacing a we have link variables
\begin{equation}
  U_{x,\mu}= e^{igaA_\mu(x)}.
\end{equation}
Here $g$ is the charge, $a$ is the lattice spacing, and $A_\mu$ is the vector potential along the link.
Note that the link variable is gauge invariant under the transformation
\begin{equation}
  U_{x,\mu}\rightarrow \Omega_x U_{x,\mu} \Omega_{x+a\hat{\mu}}.
\end{equation}
Although not necessary for the abelian case, considering $A_\mu(x)$ to have an underlying group structure will help us generalize this discussion at the end.
Therefore using the identity $\mbox{exp}(A)\mbox{exp}(B)=\mbox{exp}(A+B+\frac{1}{2}[A,B]+...)$ for matrices A and B, we can now construct the plaquette
\begin{equation}
  \begin{aligned}
    P_{x,\mu\nu}=&e^{ig}\mbox{exp}\Big[aA_\mu(x)+aA_\nu(x+\hat{\mu})-aA_\mu(x+\hat{\nu})-aA_\nu(x) \\
                 &-\frac{a^2}{2}[A_\mu(x),A_\nu(x+\hat{\mu})]-\frac{a^2}{2}[A_\mu(x+\hat{\nu}),A_\nu(x)] \\
                 &+\frac{a^2}{2}[A_\nu(x+\hat{\mu}),A_\mu(x+\hat{\nu})] + \frac{a^2}{2}[A_\mu(x),A_\nu(x)] \\
                 &+\frac{a^2}{2}[A_\mu(x),A_\mu(x+\hat{\nu})]+\frac{a^2}{2}[A_\nu(x+\hat{\mu}),A_\nu(x)]+O(a^3)\Big]
  \end{aligned}
\end{equation}
Taylor expanding the shifted terms yields
\begin{equation}
  A_\nu(x+\hat{\mu})=A_\nu(x)+a\partial_\mu A_\nu(n)+O(a^2),
\end{equation}
so that the plaquette becomes
\begin{equation}
  \begin{aligned}
  P_{x,\mu\nu}&=e^{ig}\mbox{exp}\Big[a^2\big(\partial_\mu A_\nu(x)-\partial_\nu A(x) + [A_\mu(x),A_\nu(x)]\big)+O(a^3)\Big] \\
              &=\mbox{exp}\Big[iga^2F_{\mu\nu}(x)+O(a^3)\Big]  \\
              &=1+iga^2F_{\mu\nu}(x)-\frac{g^2a^4}{2}F_{\mu\nu}(x)F^{\mu\nu}(x) + O(a^2)
  \end{aligned}
\end{equation}
In the last step we Taylor expand about $a$.
We can take the trace, note that $F_{\mu\nu}$ is traceless,
\begin{equation}
  \mbox{Tr}[P_{x,\mu\nu}]=\mbox{Tr}\Big[1-\frac{g^2a^4}{2}F_{\mu\nu}(x)F^{\mu\nu}(x) + O(a^2)\Big].
\end{equation}
From this relationship it is clear that we can construct the following action from the plaquette
\begin{equation}
  S_G(P_{x,\mu\nu})=\frac{2}{g^2}\sum_{x} \sum_{\mu<\nu} \big(1-P_{x,\mu\nu}\big).
\end{equation}
As we take the continuum limit, $a\rightarrow 0$, we recover the continuum action
\begin{equation}
  S\rightarrow \int d^4x\frac{1}{4}F_{\mu\nu}F^{\mu\nu}.
\end{equation}
This can be trivially generalized to $SU(N_c)$ gauge theories
\begin{equation}
  S_G(P_{x,\mu\nu})=\beta\sum_{x} \sum_{\mu<\nu} \mbox{Re}\Big[\mbox{Tr}\big[\mathbb{I}-\frac{1}{N_c}P_{x,\mu\nu}\big]\Big],
\end{equation}
where $\beta\equiv\frac{2N_c}{g^2}$.
This is the Wilson gauge action \cite{PhysRevD.10.2445}.

There are several ways to improve upon the Wilson gauge action.
The general idea is to add additional gauge invariant terms into the action with the goal of removing higher order lattice discretization errors \cite{Symanzik1983205, Weisz19831, Lüscher1985250}.
There are many improved actions that have been developed in the past few decades.
Although improved gauge actions in general are beyond the scope of this thesis, I will briefly comment on a negative adjoint term that we use in our action.

\subsection{Negative Adjoint Term}

In our studies we have added a negative adjoint term to the gauge part of the action.
The purpose of this term is to avoid a well known spurious UV fixed point \cite{PhysRevD.24.3212,Bhanot1982337,Blum1995301} in the fundamental adjoint plane.
The mixed fundamental-adjoint action is
\begin{equation}
  S_{AF}=\beta_F\sum_{x} \sum_{\mu<\nu} \mbox{Re}\Big[\mbox{Tr}\big[\mathbb{I}-\frac{1}{N_c}P_{x,\mu\nu}\big]\Big] + \beta_A\sum_{x} \sum_{\mu<\nu}\Big[\mbox{Tr}\big[\mathbb{I}-\frac{1}{N_c^2}P_{x,\mu\nu}^\dagger P_{x,\mu\nu}\big]\Big]
\end{equation}
In our studies we set the ratio $\frac{\beta_A}{\beta_F}=-0.25$ which avoids a crossover that extends from the UV fixed point across the $\beta_A=0$ axis \cite{Hasenfratz:2010fi}.
With our action the perturbative relation to the bare coupling is
\begin{equation}
  \frac{6}{g^2}=\beta_F+2\beta_A.
\end{equation}

% 2.4
% Composite Higgs

One solution to the hierarchy problem is for the Higgs to be a composite composed of particles from a new strongly interacting sector \cite{Kaplan:1983sm,Georgi:1984af,Weinberg:1975gm,Susskind:1978ms, Weinberg:1979bn, Eichten:1979ah, Dimopoulos:1979es}.
This new sector is responsible for electroweak symmetry breaking.
There are many types of theories that use such a modus operandi.
While these theories have the benefits of solving the heirarchy and fine tuning problems with the standard model Higgs they typically favor a heavier Higgs mass than what has been observed and pose other theoretical challenges that I will elaborate on below.
In the next three subsections I will briefly discuss technicolor, extended techniclolor, introduce the conformal window, and discuss the state of Lattice endeavors in this area.

\subsection{Technicolor}

Technicolor seeks to replace the Higgs sector of the electroweak theory with a new strongly interacting sector.
In this section I will show how this is accomplished.
Some good reviews on the subject can be found in references \cite{Hill2003235,lane,Shrock:2007km,martin,Sannino:2009za}
For simplicity lets consider a $SU(3)_{TC}$ gauge group that couples to a pair of massless fermions $\Psi=(\psi_1,\psi_2)$ in the fundamental representation.
It is worth noting that so far this theory is identical to QCD with massless up and down quarks.
The Lagrangian for our theory is
\begin{equation}
  \begin{aligned}
    \Lagr_{TC}=-\frac{1}{4}F^a_{\mu\nu}F^{\mu\nu,a}+\sum_i \bar{\Psi}(i\slashed{D})\Psi_i.
  \end{aligned}
\end{equation}
Recall that fermions have a right handed and left handed component $\phi=(\phi_L,\phi_R)$.
We know from QCD that such a theory possesses a global $SU(2)_L\times SU(2)_R$ symmetry and that this symmetry is spontaneously broken to the vector subgroup $SU(2)_V$.
This symmetry manifests itself in the nonzero VEV of the chiral condensate
\begin{equation}
  <\bar{\Psi}\Psi>=<\bar{\Psi}_L\Psi_R+\bar{\Psi}_R\Psi_L>\neq 0.
\end{equation}

To see how spontaneous chiral symmetry breaking translates to spontaneous electroweak symmetry breaking we have to use chiral perturbation theory \cite{Gasser:1983yg,Golterman:2009kw}.
Chiraly perturbation theory is an effective field theory whose fundamental degrees of freedom are the Goldstone bosons associated with the symmetry breaking.
In our case the broken $SU(2)$ symmetry has 3 degrees of freedom and therefore our effective theory will have three massless Goldstone bosons.
Continuing our analogy to QCD, these bosons are the pions.

The chiral Lagrangian is nonlinear and possesses an infinite number of terms.
The standard approach is to perform an expansion about momentum, $p$, that are small with respect to the cutoff $\Lambda$.
We are then free to choose what order in $\frac{p}{\Lambda}$ we work with.
The lowest order in the expansion is
\begin{equation}
  \Lagr_{\chi}=\frac{F^2}{4}Tr\Big[(D^\mu U)^\dagger(D_\mu U)\Big].
\end{equation}
U is a non-linear function of the Goldstone fields $\phi_a$
\begin{equation}
  U\equiv e^{\sigma^a\phi_a\frac{2i}{f}}.
\end{equation}

The derivative $D_\mu$ is coupled to the electroweak and gauge field because our technifermions are charged under $SU(2)\times U(1)$
\begin{equation}
  D_\mu=\partial_\mu - ig\frac{\sigma^a}{2}A^a_\mu+ig'\frac{\sigma^3}{2}B_\mu.
\end{equation}
We can substitute our derivative into our lowest order effective Lagrangian and find that\begin{equation}
  \Lagr_{\chi}=\frac{F^2}{4}Tr\left[\frac{g^2}{4}\left[\left(A^1_\mu-\frac{4}{Fg}\partial_\mu\phi_1\right)^2 + \left(A^2_\mu-\frac{4}{fg}\partial_\mu\phi_2\right)^2 + 
  \left(A^3_\mu-\frac{g'}{g}B_\mu-\frac{4}{fg}\partial_\mu\phi_3\right)^2\right]\right].
\end{equation}
We can now make the field redefinitions
\begin{equation}
  \begin{aligned}
    W^{1,2}_\mu &\equiv A^{1,2}_\mu - \frac{4}{fg}\partial_\mu\phi_{1,2} \\
    Z_\mu &\equiv \frac{g}{\sqrt{g^2+g'^2}}\left(A^3_\mu - \frac{g'}{g}B_\mu - \frac{4}{fg}\partial_\mu\phi_3\right) \\
    A_\mu &\equiv \frac{g}{\sqrt{g^2+g'^2}}\left(\frac{g'}{g}A^3_\mu+B_\mu\right).
  \end{aligned}
\end{equation}
The field redefinition completely removes $\phi_a$ from our theory to first order.
The field $A_\mu$ is the massless photon field.
Our Lagrangian now reads
\begin{equation}
  \Lagr_{\chi}=\Lagr_{\mbox{kinetic}}+\frac{1}{2}m_W^2\left[(W^1_\mu)^2+(W_\mu^2)^2\right]+\frac{1}{2}m_Z^2Z^2_\mu,
\end{equation}
where the masses for the heavy bosons are
\begin{equation}
  \begin{aligned}
    m_W&=\frac{1}{2}Fg \\
    m_Z&=\frac{1}{2}F\sqrt{g^2+g'^2}.
  \end{aligned}
\end{equation}
With the substitution $F=\nu$ we reproduce $m_W$ and $m_Z$ that we found in section \ref{sec:higgs_mech} exactly!
This is quite a shock, as I mentioned earlier so far everything we are doing is simply QCD with 2 flavors of massless quarks, why did we introduce the Higgs field in the first place?
Simply put the effect in QCD is much too small.
In QCD, $F=F_\pi\approx93MeV$, this is more than a factor of a thousand too small to account for the observed $W^\pm$ and $Z$ masses.

Despite the fact that spontaneous chiral symmetry breaking in QCD is too small to account for the observed electroweak sector, there is no reason that we can't introduce a new strongly interacting theory where the equivalent $F=\nu\approx246$.
Furthermore we can choose a different gauge group, number of fermions, and representation if we wish.
The only constraint is that chiral symmetry is broken and the results are phenomenologically consistent with experimental observation.
It is worth noting that more flavors of fermions will produce additional pions that will acquire masses.
In general $SU(N_f)_L\times SU(N_f)_R\rightarrow SU(N_f)_V$ will produce $N^2_f-1$ Goldstone bosons, three can be eaten to form the $W^\pm$ and $Z$ and the remaining $N_f^2-4$ become massive pseudo-Goldstone bosons.
The masses of the pseudo-Goldstone bosons in a viable theory need to be large enough to not have been discovered yet.

In summary we have shown how strong dynamics can explain electroweak symmetry breaking.
Because of asymptotic freedom, such a theory does not have UV divergences, solving the fine tuning problem.
One failure of our model is that we have lost an explanation for the fermion masses.
It is possible to construct an effective theory of fermion masses using a four fermion operator and adjusting the couplings for each quark and lepton flavor by hand.
Unfortunately a four-Fermi interaction is non-renormalizable, for a UV complete technicolor that gives fermion masses we need extended Technicolor.

\subsection{Extended Technicolor}

In order for a theory of electroweak symmetry breaking to replace the Higgs mechanism it must also generate masses for the quarks and leptons of the standard model.
This is accomplished through a framework called extended technicolor (ETC)\footnote{The first rule of ETC is no scalar fields, the second rule of ETC is no scalar fields.}. 
Under this framework an extended technicolor group is introduced $SU(3+N_{TC})_{ETC}$.
This extended group is also asymptotically free and therefore UV complete.
Both the usual standard model content as well as $SU(N_{TC})$ are charged under ETC.
At some scale $\Lambda_{ETC}>>\nu$ the ETC group spontaneously breaks to $SU(N)_{TC}\times SU(3)_c\times SU(2)_L\times U(1)_Y$.
The gauge bosons of ETC theories have masses $M_{ETC} \approx g_{ETC}\Lambda_{ETC}$.
Because of the wide range of masses for standard model fermions, most models ETC models spontaneously break at multiple scales.
Most models have three breaking scales, corresponding to the three generations of quarks and leptons.
ETC is an ambitions theory, a successful model of ETC will solve almost all the outstanding theoretical issues with the SM. 
Both the flavor hierarchy problem and the number of free parameters in the standard model would be solved dynamically.
It is not surprising then that no `reasonable' ETC model has been produced yet.

Extended technicolor faces many practical problems.
One major hurdle is flavor changing neutral currents (FCNC).
Flavor changing neutral currents such as $\mu\rightarrow e\bar{e}e$ and mixing between neutral mesons are highly constrained in the standard model.
Most ETC theories typically introduce FCNC that are proportional to $M^2_{ETC}$.

Another issue with ETC theories is size of the top quark mass.
The top quark is so massive, $m_t=172 GeV$, that its associated ETC scale is $M_{ETC}\approx 3 TeV$.
This is comparable to the electroweak scale!
Additionally incorporating both the top and the bottom with the same ETC breaking would require an accompanying large isospin breaking.
This has conundrum has forced many models to give the top special treatment in so called top assisted ETC.

Skirting these and other issues without producing particles light enough to have already been discovered is a daunting task.
There are many models that have been able to trade one defect for another, but to date no complete model exists.
There is a general consensus that any viable ETC model probably will have a slowly running (walking) coupling.
In a walking theory $\gamma(\alpha(\mu))$ is large between $\Lambda_{TC}$ and $M_{ETC}$ 
It has been shown that walking behavior addresses all of the problems I have discussed to a point.
However any model that breaks at multiple scales and has the peculiarities of walking between those scales is likely to be very baroque.
I discuss how walking a walking theory can be generated in more detail below.

\subsection{Conformal Window}
\label{sec:sec:conformalwindow}
A general class of strongly interacting theories that are of interest for Technicolor and extended Technicolor dynamics are Yang-Mills gauge theories \cite{DeGrand:2010ba,DeGrand:2009mt}.
Pure gauge Yang-Mills theories have $SU(N_C)$ interactions while more general theories include $N_f$ flavors of fermions in some representation.
At low energies QCD is effectively a $SU(3)$ gauge theory with $N_f$=2.
This description includes only the up and down quarks which are nearly massless and respect an approximate $SU(2)$ isospin symmetry.
The most general Yang-Mills Lagrangian with a $SU(N_c)$ local gauge symmetry and $N_f$ flavors of massless fermions in a representation R is:
\begin{equation}
  \Lagr_{YM}=-\frac{1}{4g^2}\sum\limits_{a=1}^{N_c} F^a_{\mu\nu}F^{\mu\nu,a}+\sum\limits_{i=1}^{N_f} \bar\Psi_i\left(i\slashed{D}\right)\Psi_i.  
\end{equation}
$F^a_{\mu\nu}$ is the field strength tensor, shown here for an arbitrary group
\begin{equation}
  F^a_{\mu\nu}=\partial_\mu A^a_\nu-\partial_\nu A^a_\mu + gf^{ijk}A^i_\mu A^j_\nu,
\end{equation}
where $A^a_\mu$ is the field, a is a group index and the structure coefficient $f^{ijk}$ is determined by the commutation relationship
\begin{equation}
  [T^i,T^j]=if^{ijk}T^k,
\end{equation}
and $T^a$ are the generators of the group.
The Lagrangians may look similar however their behavior in both UV and IR can be dramatically different from theory to theory.
While the Lagrangian is classically scale invariant; the quantum theory is not.
This is understood through the beta function,
\begin{equation}
  \beta (g^2)=-\mu\frac{\partial g^2}{\partial \mu}.
\end{equation}
The beta function describes how the gauge coupling evolves as the renormalization scale $\mu$ is changed.
This function can be expanded in perturbation theory and is universal to two loops:
\begin{equation}
  \beta\left(g^2\right)=-\frac{b_1}{16\pi^2}g^4-\frac{b_2}{(16\pi^2)^2}g^6
\end{equation}
Any terms beyond two loops are renormalization scheme dependant and are not relevant for the following discussion.
The values for $b_1$ and $b_2$ are:
\begin{equation}
  \label{eqn:b1b2}
  \begin{aligned}
    b_1&=\frac{11}{3}N_c-\frac{4}{3}N_fT(R)\\
    b_2&=\frac{34}{3}N_c^2-\frac{4}{3}T(R)N_f\Big[5N_c+3C_2(R)\Big].
  \end{aligned}
\end{equation}
$T(R)$ and $C_2(R)$ are the first and second Casimir invariants and depend on the representation R of the group.
Table \ref{table:casimir} give these invariants for a few common representations.

\begin{table}
  \centering
  \begin{tabular}{c|ccc}
    Representation    & dim($R$)            & $T(R)$          & $C_2(R)$                \\
    \hline\\
    $F$               & $N$                 & $\frac{1}{2}$   & $\frac{N^2-1}{2N}$      \\\\
    $S_2$             & $\frac{N(N+1)}{2}$  & $\frac{N+2}{2}$ & $\frac{(N+2)(N-1)}{N}$  \\\\
    $A_2$             & $\frac{N(N-1)}{2}$  & $\frac{N-2}{2}$ & $\frac{(N-2)(N+1)}{N}$  \\\\
    $G$               & $N^2-1$             & $N$             & $N$                     \\\\
  \end{tabular}
  \label{table:casimir}
  \caption{This table summarizes some commonly used groups, their dimension, first Casimir invariant, and second Casimir invariant.  $F, S_2, A_2$ and $G$ are the is the fundamental, 2-index symmetric, 2-index antisymmetric, and adjoint representations respectively.}
\end{table}

\begin{figure}[h]
  \centering
  \includegraphics[height=3in]{\cnum{2}/fig/confining.png}
  \caption{In a confining theory, such as QCD, the running of the coupling starts out negative.  As $g^2$ grows the running becomes increasingly negative.  Chiral symmetry is broken and the theory is described by confined bound states.}
  \label{fig:confining}
\end{figure}

Clearly, tuning $N_c$, $N_f$ and $R$ offers a great deal of freedom in specifying the gauge theory.
We can see from equation \ref{eqn:b1b2} that if the coefficients $b_1$ and $b_2$ are both positive than the beta function is negative.
In this scenario the theory asymptotically free, confining, and spontaneously breaks chiral symmetry.
The dynamics in the IR will be strongly coupled and non-perturbative.
Figure \ref{fig:confining} shows what this scenario looks like.
This occurs in QCD ($SU(3)$ $N_f=2\mbox{ or }3$ depending on your treatment of the light quarks), which is an example of a theory where the beta function is negative.

\begin{figure}[h]
  \centering
  \includegraphics[height=3in]{\cnum{2}/fig/conformal.png}
  \caption{In a conformal field theory the $\beta$ function starts out negative but turns around and crosses the $\beta=0$ axis.  At this crossing the theory has an infrared fixed point.  The renormalized flow for the theory will run to the IR fixed point and the theory is conformal.}
  \label{fig:conformal}
\end{figure}

If we keep $b_1$ positive and allow $b_2$ to become negative we can force the two terms to compete.
In perturbation theory this occurs at at a critical value of fermions $N^{crit}_f<\frac{17N_c^2}{2T(R)[5N_c+C(R)]}$.
Such a beta function would start out negative, then at some coupling it would pass through a local minimum after which it would grow.
Eventually, at some $g^2_*$, the beta function would have a zero where $\beta(g^2_*)=0$.
Perturbatively this zero occurs at $g^2_*=-b_1/b_2$ and is a Banks-Zaks infrared fixed point.
This is illustrated in figure \ref{fig:conformal}, such a theory is governed by the conformal dynamics at the infrared fixed point and is scale invariant.
If the fixed point is at very weak coupling it is possible to gain insights from perturbation theory, however many theories are known to have strongly coupled IRFP's where insights from perturbation theory are unreliable.
Conformal theories, like the scenario described above, do not support bound states of particles.
As such they are often referred to as `unparticle theories'.

\begin{figure}[h]
  \centering
  \includegraphics[height=3in]{\cnum{2}/fig/trivial.png}
  \caption{If the fermionic degrees of freedom overwhelm those of the bosons it the beta function will start positive.  This is the loss of asymptotic freedom.  The beta function for such a theory resembles that of QED.}
  \label{fig:trivial}
\end{figure}

Allowing both $b_1$ and $b_2$ to be negative results in a trivial theory.
Such a theory, shown in figure \ref{fig:trivial} is not asymptotically free.  
The beta function is very similar to that of QED.
Perturbatively this occurs when $N_f<\frac{11N_c}{4T(R)}$.

The region in theory space between where the IRFP appears and where asymptotic freedom is lost is referred to as the conformal window.
Limiting our consideration to theories using the fundamental representation fermions and $N_c=3$ we see that perturbation theory predicts the conformal window is between $N_f \approx 9.4 \mbox{ and } N_f=16.5$.
It is important to note that all of the discussion in this section are perturbative results.
On the upper end of the conformal window before asymptotic freedom is lost, the IRFP of many theories is weak enough that perturbation theory is reasonable.
However, as we pass through the conformal window and consider theories with less fermionic degrees of freedom, the IRFP becomes strongly coupled.
Here perturbation theory is not to be trusted.
The location of the lower bound of the conformal window is an area of active research.

\begin{figure}[h]
  \centering
  \includegraphics[height=3in]{\cnum{2}/fig/walking.png}
  \caption{In a walking theory the beta function looks similar to the conformal picture of figure \ref{fig:conformal}.  The crucial difference is that before the $\beta$ function develops a zero, shown by the dashed line, chiral symmetry is spontaneously broken driving the beta function back down.  Near the would be IRFP where the dashed line has a zero, the coupling runs very slowly and is said to be walking.}
  \label{fig:walking}
\end{figure}

The conformal window itself is not a very interesting system to study but understanding the lower bound of the conformal window is of great interest for potential Technicolor theories.
Right below the conformal window it is believed that a walking theory can exist.
In such a theory, shown in figure \ref{fig:walking} the beta function would start out like the conformal scenario.
However, as the beta function approaches the IRFP, chiral symmetry is spontaneously broken and the beta function would turn around away from the `would be' IRFP.
If this occurs, the coupling $g$ will run very slowly (walk) over a wide range of scales where the $\beta$ function approaches zero.
Walking is necessary to achieve a wide separation of scales in Technicolor theories without generating flavor changing neutral currents.

\subsection{Lattice Studies}
Today there is a rich ecosystem of lattice studies of BSM studies, a survey of the current state of the field can be found in \cite{need to find a good overview}.
It is extremely important that different ways of probing for IR conformality on different lattice discretization schemes converge to common conclusions.
In general this is accomplished but there are still systems that cause debate because different groups draw different conclusions from their findings.  Below I summarize the sate of the field both in terms of calculation types and methods of theories studied.

Many studies that seek to locate the conformal window have proceeded by calculating the RG flow of the theory under consideration.
The first example of this type of study is \cite{Appelquist:2009ty}.
If an IRFP is found then the theory is conformal while a lack of an IRFP at strong coupling indicates that chiral symmetry is broken and the theory confines.
One benefit of running coupling calculations is that they are not as computationally intensive as many other alternatives.
There are multiple types of calculations that fall under this umbrella.
Schr\"odinger functional, MCRG (section \ref{ch:MCRG,ch:WMCRG}), and now the Wilson flow step scaling (section \ref{ch:wflow}) are examples.

Another common method uses lattice spectroscopy calculations.
In these methods the masses of light bound states are measured at a variety of input quark masses.
The fit of the bound state as a function of the bare quark mass will determine if the theory is best described by a chiraly broken theory or conformal one.
References \cite{Jin:2008rc,Jin:2009mc,Fodor:2010zzb} are examples of studying the conformal window using spectroscopy.
In practice these fits are extremely difficult, in particular fits for volume squeezed conformal theories has proved troublesome.

Several groups have studied the finite temperature phase diagram \cite{Iwasaki:2003de,Schaich:2012fr}.
These studies identify an IRFP with a deconfining bulk transition.
Additionally techniques using Dirac eigenvalue spectrum \cite{Cheng:2013eu} has proven useful in calculating $\gamma_m$ over a wide range of scales.

The majority of calculations have been for theories with $N_c=3$ with fermions in the fundamental representation.
This is in part because the wide availability of optimized code bases for three colors.
Within these studies staggered fermions are commonly employed because they are cheap to simulate and preserve a U(1) remnant of chrial symmetry \ref{}.
As a consequence theories with $N_f=4,8,12,16$ are most widely studied although others have been studied as well.
The consensus is that $N_f\le8$ are confining, $N_f=12$ is probably conformal, and $N_f=16$ is conformal.

There is also a grown number of calculations with $SU(2)$.
The two flavor model in the adjoint representation is widly studied and is clearly in the conformal window\cite{Catterall:2007yx,DelDebbio:2010hu,DeGrand:2011qd,DeGrand:2011vp}.
Studies with fundamental fermions exist for $N_f=2,4,6,8,10$ with the conformal window being identified as existing between $4<N_f<8$.
Studies of $SU(3)$ and $SU(4)$ in the 2-index symmetric representation also exist.

Clearly, beyond standard model physics is inherently a difficult subject to study on the lattice.
Unlike studies of QCD we don't know the answer before hand.
Taking a holistic approach and finding consensus is vital to our understanding.
Additionally, as more of these theories have been studied, vital improvements have been made.
This is true of the lattice code base, which prior to BSM studies was highly optimized and sometimes only available for QCD simulations.
This is also true for improvements in how we analyze lattice data.
I will discuss two improvements we have developed in chapters \ref{ch:WFLOW} and \ref{ch:WMCRG}.
Chapter \ref{ch:WFLOW} introduces an improvement to the gradient flow method for calculating the renormalized step scaling function.
Chapter \ref{ch:WMCRG} introduces an improvement to the traditional MCRG method of calculating the bare step scaling function.
Finally work in BSM physics on the lattice has pushed forward our knowledge of formulating lattice actions and understanding the nature of lattice artifacts.
Many of the theories that have been studied are sufficiently un-QCD like that lattice QCD intuitions are being upended with a new tribal knowledge.

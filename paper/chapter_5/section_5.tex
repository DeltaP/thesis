% 5.5
% Summary

We have considered step scaling based on the gradient flow renormalized coupling, introducing a non-perturbative $\cO(a^2)$ improvement that removes, or at least greatly reduces, leading-order cut-off effects.
This phenomenological improvement increases our control over the extrapolation to the infinite-volume continuum limit, as we demonstrated first for the case of SU(3) gauge theory with $N_f = 4$ massless staggered fermions.
Turning to $N_f = 12$, we found that the continuum limit was well defined and predicted an infrared fixed point even without improvement.
Applying our proposed improvement reinforced this conclusion by removing all observable $\cO(a^2)$ effects.
For the finite-volume gradient flow renormalization scheme defined by $c = 0.2$, we find the continuum conformal fixed point to be located at $\gstar = 6.18(20)$.

The 12-flavor system has been under investigation for some time, and other groups have studied its step scaling function~\cite{Lin:2012iw, Appelquist:2007hu, Appelquist:2009ty, Itou:2012qn}.
However, this work is the first to observe an IRFP in the infinite-volume continuum limit.
There are likely several factors contributing to this progress.
While we did not invest more computer time than other groups, we have employed a well-designed lattice action.
The adjoint plaquette term in our gauge action moves us farther away from a well-known spurious fixed point, while nHYP smearing allows us to simulate at relatively strong couplings.
The gradient flow coupling itself appears to be a significant improvement over other schemes,\footnote{C.-J.~D.~Lin has told us about dramatic improvements in auto-correlations when using the gradient flow coupling compared to the twisted Polyakov loop coupling of \refcite{Lin:2012iw}.} and our non-perturbative improvement also contributes to obtaining more reliable continuum extrapolations.

Our non-perturbative improvement is general and easy to use in other systems.
It does not rely on the lattice action or fermion discretization, though we suspect that the improvement may not be effective if there are $\cO(a)$ artifacts, e.g.\ for unimproved Wilson fermions.
Since $\cO(a)$-improved lattice actions are standard, this does not appear to be a practical limitation.
We look forward to seeing our proposal applied both to QCD and to other conformal or near-conformal systems.

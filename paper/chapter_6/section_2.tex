% 6.2
% Wilson Flow Optimized MCRG

As an alternative to optimizing the RG blocking transformation, and thus changing the renormalization scheme at each coupling $\be_F$, here we propose to use the Wilson flow to move the lattice system as close as possible to the renormalized trajectory of a fixed renormalization scheme.

The Wilson flow is a continuous smearing transformation~\cite{Narayanan:2006rf} that can be related to the \MSbar running coupling in perturbation theory~\cite{Luscher:2010iy}.
Refs.~\cite{Fodor:2012td, Fodor:2012qh} recently used the Wilson flow to compute a renormalized step-scaling function in a way similar to Schr\"odinger functional methods.
While this approach appears very promising, it is based on perturbative relations that are only fully reliable at weak coupling.
Here we do not use this perturbative connection, instead applying the Wilson flow as a continuous smearing that removes UV fluctuations.
The Wilson flow moves the system along a surface of constant lattice scale in the infinite-dimensional action-space; it is not a renormalization group transformation and does not change the IR properties of the system.

\begin{figure}[ht]
  \centering
  \includegraphics[height=3 in]{\cnum{6}/fig/wilson_flow_opt}
  \caption{In Wilson Flow MCRG we use the Wilson Flow (blue) to approach the renormalized trajectory.  An optimization step similar to that used in MCRG allows us to locate the flow time that gets us closest to the renormalized trajectory (orange). We then block our lattice using a fixed block transformation (green).}
  \label{fig:wilson_flow_opt}
\end{figure}

Our goal is to use a one-parameter Wilson flow transformation to move the lattice system as close as possible to the renormalized trajectory of our fixed RG blocking transformation.
This is shown in figure \ref{fig:wilson_flow_opt}.
We proceed by carrying out two-lattice matching after applying the Wilson flow for a flow time $t_f$ on all lattice volumes.
(The Wilson flow is run only on the unblocked lattices, not in between RG blocking steps.)
As above, since we can block our lattices only a few times, we must optimize $t_f$ by requiring that consecutive RG blocking steps yield the same $\De\be_F$, as shown in \ref{fig:t_optimization}.
As for traditional MCRG, increasing the number of blocking steps reduces the dependence on the optimization parameter; in the limit $n_b \to \infty$, our results would be independent of $t_f$.

\begin{figure}[ht]
  \centering
  \includegraphics[height=3 in]{\cnum{6}/fig/t_optimization.pdf}
  \caption{Optimization of the Wilson flow time $t_f$ with fixed $\al = 0.5$, for $\be_F = 4.5$.  The uncertainties on the data points are dominated by averaging over the different observables.}
  \label{fig:t_optimization}
\end{figure}

With Wilson-flowed MCRG we can efficiently determine bare step-scaling functions that correspond to unique RG \be functions.
The uniquness of the $\beta$ function is a result of using a fixed block transformation.
In this work our block transformation consists of nHYP smearing the unblocked lattice and then multiplying adjacent links in the same direction.

The ability to study a unique \be function opens up interesting directions for future studies.
By comparing different \be functions around the perturbative gaussian FP, we can study scaling violations in the lattice system.
In IR-conformal systems, we can investigate the scheme-dependence of the \be function near the IRFP, an issue explored in perturbation theory by \refcite{Ryttov:2012nt}.

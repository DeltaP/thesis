% 2.3
% Beyond the Standard Model

While the Standard Model is the pinnacle of our current understanding of the universe it is not the last word on the subject.
There are many phenomena that the standard model does not explain.
Furthermore there are theoretically troubling aspects to the standard model that leaves more to be desired.

Perhaps the biggest omission in the standard model is a quantum field theory of Gravity.
Our current understanding of gravity is Einstein's theory of General Relativity (GR).
GR is a thoroughly verified theory and famously explained the perihelion of mercury and the deflection of light rays massive objects.
Unfortunately, general relativity is incompatible with quantum mechanics, and to date a force carrying boson, the graviton, has eluded detection.
A related problem is that gravity is so much weaker than the other three fundamental forces.
Another unexplained phenomena is neutrino oscillations.
In nature we have observed the flavor of neutrinos to change.
These oscillations prove that the neutrino's are not massless particles as the standard model previously had us believe.
Finally we currently understand from astrophysical observations that the standard model only accounts for 4\% of the energy budget of the universe.
Dark matter, which has eluded detection, composes of another 21\% of the energy budget.
Dark Energy, which is responsible for the acceleration of the expansion of the universe, composes 75\% of our energy budget.
Taking this view the standard model actually explains a relatively small part of the universe.

In addition to not explaining the entire range of physical phenomena that we observe (or fail to observe) there are theoretical shortcomings that indicate a lack of complete understanding.
The first indication that something is awry is the number of parameters.
The standard model in its most general form has nineteen free parameters.
All of these parameters are fixed by experiment and cover a large range of scales.
We expect that a more nuanced understanding of the universe will require less parameters.

Another theoretical quirk of the standard model is the strong CP problem.
Unlike the electroweak interactions, QCD does not violate CP symmetry.
CP symmetry is the symmetry of charge conjugation (C) and parity (P).
The QCD Lagrangian allows terms that violate CP symmetry but these terms appear to be zero which is a type of fine tuning.

Finally there is the hierarchy problem.
In the standard model the coupling constants of the theory change with energy scale.
If you run the theory into the UV the coupling constants nearly converge at the GUT scale of $10^{16}$ GeV.
The standard model can be run further to the plank scale and it remains a consistent theory.
The fact remains that the standard model is an effective theory and at some point it must be cut off.
The cutoff of the theory is removed from several orders of magnitude from the weak scale.
This is a problem because the Higgs is an elementary scalar field and therefore has a relevant quartic self coupling.
The renormalized Higgs mass is
\begin{equation}
  m^2_H=m_0^2+\frac{3}{4\pi}\lambda\Lambda^2,
\end{equation}
where $m_0$ is the bare Higgs mass, $\lambda$ is the quartic coupling, and $\Lambda$ is the cutoff scale.
The renormalized mass has a quadratically divergent additive renormalization to its mass that is proportional to the cutoff.
To keep the renormalized Higgs mass at its physical value an unnatural degree of fine tuning to its bare mass, $m_0$, and quartic term, $\lambda$, is required.
A more natural solution to this problem is for $\Lambda$ to be at a scale similar to the Higgs mass.

These issues have pushed researchers to look for extensions or alternatives to the standard model that answer one or more of these unresolved questions.
One area of active research has been the hierarchy problem.
The state of affairs for almost 40 years was that we knew there had to be a Higgs mechanism to complete the standard model.
We also had a very good guess where to look for the Higgs boson because of the mass of the $W^{\pm}$ and the $Z$.
Solving the hierarchy problem and predicting the Higgs mass became a popular game, guess right and win a Nobel Prize.
Most methods of solving the hierarchy problem introduce new physics at the electroweak scale.
This moves the SM cutoff much closer to the electroweak scale, removing the need for fine tuning.
Because the Tevatron and more recently the LHC could reach energies high enough to probe the electroweak scale it was possible to build and test falsifiable theories.

Super Symmetry is one such approach that postulates a symmetry between fermions and bosons.
This symmetry results in a cancellation of the quadratic divergence that the scalar Higgs theory has.
Another approach which I will describe further in the next section is for the Higgs particle to be a composite particle.
A consequence of all of these theories is new particles.
Now we know the value of the Higgs mass.
Additionally, no other particles have yet been discovered by the LHC. 
This has proven to be a deadly combination, ruling out the simplest incarnations of these models.

% 1.2
% Organization
This thesis is organized into five chapters not including the introduction and conclusion.
Briefly, I use chapter 2 to set the stage and chapter 3 to give a flavor for lattice calculations.
Chapters 4-6 discuss my contributions, each chapter is its own calculation and reflects approximately a year worth of work.

I begin chapter 2 with a brief overview of the standard model.
The Higgs mechanism is given its own section due to its relevance for the remainder of the thesis.
I then summarized flaws in our current understanding of the standard model with emphasis given to the hierarchy problem.
Finally I discuss how a composite Higgs can resolve the hierarchy problem in a natural way.
Because a composite Higgs arises from a strongly interacting quantum field theory we cannot understand such a theory with the standard analytical tools of perturbation theory.
Fortunately there is a well developed nonperturbative approach, lattice gauge theory.
I conclude the chapter with an overview of current lattice studies.

Chapter 3 provides the reader with a brief overview of lattice gauge theory.
Like any mature discipline, this topic is a subject of many books.
My goal is not to cover this subject comprehensively; rather I want to provide the reader with a flavor of how a lattice calculation is performed and highlight areas that are useful in later sections.
To that end I develop the pure gauge action and discus the negative adjoint term that our group uses in our simulations.
I also discuss the problems associated with the discretization of fermions on the lattice.
I end the discussion of lattice fermions with a description of staggered fermions and the nHYP smearing we use in our calculations.

Chapters 4 - 6 document the body of work that is this thesis.
Chapter 4 concerns my early work using Monte Carlo Renormalization Group (MCRG) on SU(3) $N_f$ = 8 and SU(3) $N_f$ = 12 theories.
In chapter 5, I introduce the gradient flow and how it can be used to calculate the renormalized step scaling function.
I also use this chapter to discuss an optimization improvement that our group developed.
Results for for SU(3) $N_f$ = 4 and SU(3) $N_f$ = 12 are given.
Finally chapter 6 shows how the Wilson Flow can be used as an optimization step for MCRG.
Results from SU(3) $N_f$ = 12 are given.

In my concluding chapter I make some final remarks about interpreting our results.
I will also summarize results from other methods that our group used to study this system.
All of the methods our group has used to study there systems have converged to the same answer making a strong case that SU(3) gauge theories with 12 flavors of fundamental fermions has an infrared fixed point.

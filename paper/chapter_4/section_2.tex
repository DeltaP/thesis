% 4.2
% Method

\subsection{Blocking}
In a lattice simulation the quantities stored in the computer are the link variables.
The link variable $U_{n,\mu}$ encodes the gauge field at sight $n$ in all four $\hat{\mu}$ directions.
A MCRG blocking step is a transformation that takes two links $U_{n,\mu}$ and $U_{n+\hat{\mu},\mu}$ and creates one link with twice the lattice spacing of the initial two links.
\begin{figure}[h]
  \centering
  \includegraphics[height=3in]{\cnum{4}/fig/block2.png}
  \caption{}
  \label{fig:block2}
\end{figure}
The blocked lattice will have a factor of $2^N$ less links than the original lattice.
In four dimensions this means that an observable such as a Wilson loop will have a factor of 16 less statistics on the blocked lattice compared to the original lattice.
Fortunately, on an $N$ dimensional hyper cubic lattice there are $2^N$ unique ways to perform this blocking.
Therefore in four dimensions there are 16 unique blocking configurations.
We can preserve statistics in our calculation by performing all of the unique blockings and averaging all of the observalbes from those blockings together.
Additionally we store the blocked links on a lattice of the original cardinality using offsets as shown in \ref{fig:block2}.

The blocked links have a lattice spacking of twice the original lattice, $a' = 2a$ such that the physical size of the box has not been changed.


\subsection{Two-lattice matching procedures and the need for optimization} % Draft complete
Two-lattice matching is most easily described in the context of confining systems, where it locates pairs of couplings $(\be_F, \be_F')$ for which lattice correlation lengths obey $\xi(\be_F) = 2\xi(\be_F')$.
We proceed by repeatedly applying RG blocking transformations (with scale factor $s = 2$) to lattices of volume $24^3\X48$, $12^3\X24$ and $6^3\X12$.\footnote{We are currently generating larger lattices up to $32^3\X64$, which will permit additional consistency checks.}
Under RG blocking on the $m = 0$ critical surface, the system flows toward the renormalized trajectory in irrelevant directions, and along it in the relevant direction.
By blocking the larger lattices (with $\be_F$) $n_b$ times and the smaller lattices (with $\be_F'$) only $n_b - 1$ times, we obtain blocked systems with the same lattice volume.
If these blocked systems have both flowed to the same point on the renormalized trajectory, then we can conclude that $\xi(\be_F) = 2\xi(\be_F')$ on the unblocked systems, as desired.

We determine whether the blocked systems have flowed to the same point on the renormalized trajectory by matching several short-range gauge observables: the plaquette, all three six-link loops, and two planar eight-link loops.
For a given $\be_F$, each observable may predict a different $\De\be_F \equiv \be_F - \be_F'$.
The spread in these results is a systematic error that dominates our uncertainties.

In an IR-conformal system, the gauge coupling that is relevant at the perturbative gaussian FP becomes irrelevant at the IRFP.
The renormalized trajectory connects these two fixed points.
When RG flows approach this renormalized trajectory, the two-lattice matching can be performed and interpreted the same way as in confining systems.
In this region the gauge coupling flows to stronger couplings, $\De\be_F > 0$ corresponding to a negative RG \be function.
The situation is less clear at stronger couplings where we might na\"ively expect backward flow.
If there is no ultraviolet FP in this region to drive the RG flow along a renormalized trajectory, the two-lattice matching might become meaningless.
This issue affects every method that attempts to determine the flow of the gauge coupling in IR-conformal systems at strong coupling.
In all published studies that report an IRFP, backward flow has only been observed in a very limited range of couplings in the immediate vicinity of the IRFP (cf.~\refcite{Giedt:2012LAT}).

Since we can block our lattices only a few times, we must optimize the two-lattice matching by requiring that consecutive RG blocking steps yield the same $\De\be_F$.
We identify the optimized $\De\be_F$ with the bare step-scaling function $s_b$.
In the following subsections, we describe two different ways to perform this optimization.
The traditional technique optimizes the RG blocking transformation (renormalization scheme).
The new method we propose in \secref{sec:WMCRG} instead applies the Wilson flow to the lattice system prior to RG blocking, and optimizes the flow time $t_f$.

%First, we note that finite-volume effects are minimized by carrying out the optimization on blocked lattices of the same volume, which was reported by \refcite{Hasenfratz:2011xn}.
%That is, we should compare $\De\be_F$ from matching $12^3\X24$ blocked to $3^3\X6$ vs.\ $6^3\X12$ blocked to $3^3\X6$ with that from matching $24^3\X48$ blocked to $3^3\X6$ vs.\ $12^3\X24$ blocked to $3^3\X6$.
%However, the error introduced by matching $24^3\X48$ blocked to $6^3\X12$ vs.\ $12^3\X24$ blocked to $3^3\X6$ instead of generating $6^3\X12$ lattices is not severe, and our preliminary results in \secref{sec:WMCRGresults} will omit the finite-volume correction.
% ------------------------------------------------------------------



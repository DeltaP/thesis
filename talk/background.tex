
                      %******************%
%********************% Background         %********************%
                      %******************%

\section{Background}
\addtocounter{framenumber}{-1}

\begin{frame}
  \frametitle{The Standard Model}
  \begin{columns}[T]
    \begin{column}{.6\textwidth}
      \includegraphics[width=\textwidth]{fig/higgs_standard_mod.png}
    \end{column}
    \begin{column}{.4\textwidth}
      Explained:
      \begin{itemize}
        \item Strong Nuclear Force
        \item Weak Nuclear Force
        \item Electricity and Magnetism
        \item Electroweak Symmetry Breaking
      \end{itemize}
      Unexplained:
      \begin{itemize}
        \item Gravity
        \item Dark Matter
        \item Dark Energy
        \item Neutrino Masses
      \end{itemize}
    \end{column}
  \end{columns}
  \emph{Scalar Higgs $\rightarrow$ Hierarchy Problem!}
\end{frame}

\begin{frame}
  \frametitle{Solving the Hierarchy Problem with a Composite Higgs}
  \begin{columns}[T]
    \begin{column}{.5\textwidth}
      Hierarchy Problem:
      \begin{itemize}
        \item $m^2_H=m^2_0+\frac{3}{4\pi}\lambda\Lambda_{cut}^2$
        \item $m_H=126$ GeV
        \item $\Lambda_{cut} \gg m_H$
        \item tune $m_0$, $\lambda$ to balance $\Lambda_{cut}$
      \end{itemize}
    \end{column}
    \begin{column}{.5\textwidth}
      A Possible Solution:
      \begin{itemize}
        \item Asymptotically free, gauge-fermion QFT 
        \item Goldstone bosons from the broken chiral symmetry give $W^{\pm}$ and $Z$ mass
        \item Higgs:  $0^{++}$
        \item Higgs must be light (126 GeV)
        \item Other bound states must be heavy
      \end{itemize}
    \end{column}
  \end{columns}
  \vspace{12pt}
  \emph{Any viable theory needs to have a slowly running coupling (walking) constant.}
\end{frame}

\begin{frame}
  \frametitle{Walking coupling}
  \begin{center}
    \includegraphics[width=\textwidth]{fig/running_cartoon.png}
  \end{center}
  \vspace{12pt}
  \emph{How the coupling changes over energy scale is determined by the $\beta$ function.}
\end{frame}

\begin{frame}
  \frametitle{Hints from the Perturbative $\beta$ Function}
  The $\beta$ function encodes how the coupling, $g$, changes with respect to scale, $\mu$.
  The 2-loop $\beta$ function is scheme independent.
  \begin{center}
    \begin{align*}
      \beta(g^2)&=\frac{d g^2}{d\text{log}(\mu^2)}=-\frac{b_1}{16\pi^2}g^4-\frac{b_2}{(16\pi^2)^2}g^6\\
      b_1&=\frac{11}{3}N_c-\frac{4}{3}N_fT(R)\\
      b_2&=\frac{34}{3}N_c^2-\frac{4}{3}T(R)N_f\Big[5N_c+3C_2(R)\Big]
    \end{align*}
  \end{center}
  \vspace{12pt}
  \begin{columns}[T]
    \begin{column}{.5\textwidth}
      \small
      \begin{tabular}{l l}
        $N_c$: & Number of Colors\\
        $N_f$: & Number of flavors of fermions\\
        $R$:   & Representation of the fermions
      \end{tabular}
    \end{column}
    \begin{column}{.5\textwidth}
      \small
      \begin{tabular}{l l}
        $T(R)$: & First Casimir invariant\\
        $C_2(R)$: & Second Casimir invariant
      \end{tabular}
    \end{column}
  \end{columns}
\end{frame}

\begin{frame}
  \frametitle{The $\beta$ Function in Pictures}
  \begin{equation*}
    \beta(g^2)=\frac{d g^2}{d\text{log}(\mu^2)}=-\frac{b_1}{16\pi^2}g^4-\frac{b_2}{(16\pi^2)^2}g^6
  \end{equation*}
  \centering
  \includegraphics[width=0.7\textwidth]{fig/betafn.png}
\end{frame}

\begin{frame}
  \frametitle{Roadmap for the Conformal Window}
  \begin{columns}[T]
    \begin{column}{.6\textwidth}
      \includegraphics[width=1.1\textwidth]{fig/lattice_studies.pdf}
    \end{column}
    \begin{column}{.4\textwidth}
      \small
      Conformal Window:
      \begin{tabular}{r l}
        Shaded: & Schwinger-Dyson\\
        Dashed: & 2-loop $\beta$ function
      \end{tabular}
      \vspace{24pt}
      \newline
      Fermion Representation:
      \begin{tabular}{r l}
        F: & Fundamental\\
        2A: & 2-index Antisymmetric\\
        2S: & 2-index Symmetric\\
        Adj: & Adjoint
      \end{tabular}
    \end{column}
  \end{columns}
  \vspace{12pt}
  The exact location of the conformal window requires non-perturbative investigation.
\end{frame}

\begin{frame}
  \frametitle{Studying the $\beta$ Function on the Lattice}
  \begin{center}
    There are several techniques:
  \end{center}
  \begin{itemize}
    \item Schrodinger Functional
    \item Twisted Polyakov Loop scheme
    \item \textbf{Monte Carlo Renormalization Group (MCRG)}
    \item \textbf{Wilson Flow Step Scaling}
    \item Wilson Flow Step Scaling + Schrodinger Functional
    \item \textbf{Wilson Flow MCRG}
  \end{itemize}
\end{frame}

\begin{frame}
  \frametitle{Simulation Details}
  \begin{itemize}
    \item Staggered Fermions
    \item nHYP smearing
    \item Negative adjoint gauge term with $\beta_A/\beta_F=-0.25$
    \item hybrid Monte Carlo (HMC algorithm)
    \item code based on MILC collaboration code
  \end{itemize}
\end{frame}

% 3.3
% Fermionic Fields
\subsection{Naive Fermions}
Fermions prove to be a much thornier issue and great care must be taken with their discretization.
Lets see what happens when we take a theory of one free fermion
\begin{equation}
  S_f=\int d^4x\Big(\bar{\psi}(x)\gamma^\mu\partial_\mu\psi(x)+m\bar{\psi}(x)\psi(x)\Big),
\end{equation}
and define the discretized lattice derivative to be the difference 
\begin{equation}
  \partial_\mu\psi(x)\rightarrow\Delta_\mu\psi_x=\frac{\psi_{x+\hat{\mu}}-\psi_{x-\hat{\mu}}}{2a}.
\end{equation}
The fermionic fields $\psi$ have the following gauge transformation:
\begin{equation}
  \begin{aligned}
    \psi_x&\rightarrow\Omega(x)\psi_x \\
    \bar{\psi}_x&\rightarrow\bar{\psi}_x\Omega^\dagger(x).
  \end{aligned}
\end{equation}
We will now use the link variables $U_{x,\mu}$ to transport the $\psi$ fields between neighboring sites
\begin{equation}
  S_f=\frac{1}{2}\sum_{x,\mu} \bar{\psi}_x\gamma^\mu\big[U_{x,\mu}\psi_{x+\hat{\mu}}-U^\dagger_{x-\hat{\mu}}\psi_{x-\hat{\mu}}\big]+m\sum_x \bar{\psi}_x\psi_x.
\end{equation}
Now we can construct the propagator in momentum space
\begin{equation}
  S(p)=\frac{m-i\sum_{\mu}\gamma_\mu\sin(p_\mu a)/a}{m^2+\sum_\mu \sin^2(p_\mu a)/a^2},
\end{equation}
where the momenta are restricted by the lattice discretization $p_\mu=\frac{2\pi ix_\mu}{N_\mu}$ for a lattice of $N_\mu$ sites in direction $\mu$.
For small momenta, $p_\mu \approx (0,0,0,0)$ we can Taylor expand $\gamma^\mu\sin(p_\mu)\approx \slashed{p}$ and recover the continuum fermionic propagator $\frac{1}{\slashed{p}-m}$.
Unfortunately we also recover the continuum propagator for fifteen other values of the four lattice momenta $p_\mu\approx(\pi,0,0,0),p_\mu\approx(0,\pi,,0,0),...,p_\mu\approx(\pi,\pi,\pi,\pi)$.
Thus instead of describing one species of fermion, our theory instead describes 16 degenerate species of fermions.
This is the fermion doubling problem.
For each dimension $d$ of the lattice we get $2^d$ species of fermions.

\subsection{Wilson Fermions}
Wilson solved the fermion doubling problem by adding a second derivative term to the action,
\begin{equation}
  \Delta_\mu^2\psi_x=\frac{\psi_{x+\hat{\mu}}-2\psi_x+\psi_{x-\hat{\mu}}}{2a}.
\end{equation}
Such a term does not affect our continuum extrapolation.
The resulting action
\begin{equation}
  S_f=-\frac{1}{2}\sum_{x,\mu}\bar{\psi_x}\Big[(1-\gamma^\mu)U_{x,\mu}\psi_{x+\hat{\mu}}+(1+\gamma^\mu)U^\dagger_{x-\hat{\mu}}\Big]+(4+m)\sum_x\bar{\psi}_x\psi_x,
\end{equation}
has the following free momentum space propagator
\begin{equation}
  \begin{aligned}
    \frac{1}{a}S(p)&=\Big[i\gamma^\mu\sin(p_\mu)+m+\sum_\mu(1-\cos(p_\mu))\Big]^{-1}  \\
                   &=\frac{-i\gamma^\mu\sin(p_\mu)+m+\sum_\mu\big(1-\cos(p_\mu)\big)}{\sum_\mu\sin^2(p_\mu)+\Big[m+\sum_\mu\big(1-cos(p\mu)\big)\Big]^2}.
  \end{aligned}
\end{equation}
Now all of the doubler modes with any component of momentum $p_\mu\approx\pi$ become massive.
The mass doubler modes gain is proportional to the inverse of the lattice spacing.
The caveat in this strategy is that the Wilson term explicitly breaks chiral symmetry on the lattice even at zero bare fermion mass.

Unfortunately there is no easy way around this problem.
The Nielsen and Ninomiya no-go theorem states that you cannot place chiral fermions on a regular lattice of even dimension without fermion doublers \cite{Nielsen198120,Nielsen1981173,Nielsen1981219,kaplan:fermions}.
Any solution to the doubler problem will have to sacrifice chiral symmetry, dimensionality, or the regularity of the lattice.
There are several types of fermionic actions that trade off these pitfalls.
Some actions are able to preserve a near exact chiral symmetry, however these actions can be very expensive and difficult to simulate at strong coupling.
The only action I will address further is the staggered action because it is used in our study.

\subsection{Staggered Fermions}

Staggered fermions were introduced by Kogut and Susskind \cite{PhysRevD.11.395}.
The staggered action mixes Dirac and space time indices of the naïve fermionic action to reduce the number of doublers from 16 to 4.
To accomplish this lets define new field variables $\psi(x)'$ and $\bar{\psi}(x)'$
\begin{equation}
  \begin{aligned}
    \psi(n)&=\gamma_1^{n_1}\gamma_2^{n_2}\gamma_3^{n_3}\gamma_4^{n_4}\psi(n)'\\
    \bar{\psi}(n)&=\bar{\psi}(n)'\gamma_4^{n_4}\gamma_3^{n_3}\gamma_2^{n_2}\gamma_1^{n_1}.
  \end{aligned}
\end{equation}
The transformation matrix is products of $\gamma_\mu$ raised to the power of the corresponding component $n_\mu$ of the site index $n_\mu=(n_1,n_2,n_3,n_4)$.
We can now use these fields to build the staggered action
\begin{equation}
  S(\psi',\bar{\psi}')=a^4\sum_n \bar{\psi}(x)'\mathbb{I}\Big(\sum_{\mu=1}^4\eta_\mu(x)\frac{\psi(n+\hat{\mu})'-\psi(n-\hat{\mu})'}{2a}+m\psi(n)'\Big),
\end{equation}
where $\eta_\mu$ are the staggered sign functions
\begin{equation}
  \begin{aligned}
    n_1(n)&=1 \\
    n_2(n)&=(-1)^{n_1} \\
    n_3(n)&=(-1)^{n_1+n_2} \\
    n_4(n)&=(-1)^{n_1+n_2+n_3}.
  \end{aligned}
\end{equation}
This action has four copies of itself, one for each Dirac component.
We finalize the action by taking one of these copies and coupling it to the gauge fields
\begin{equation}
  S(\chi,\bar{\chi})=a^4\sum_n \bar{\chi}(n)\Big(\sum_{\mu=1}^4\eta_\mu(x)\frac{U_\mu(n)\chi(n+\hat{\mu})-U_\mu^\dagger(n-\hat{\mu})\chi(n-\hat{\mu})}{2a}+m\chi(n)\Big).
\end{equation}
The fields $\chi$ and $\bar{\chi}$ are Grassmann fields with only color indices and lack Dirac structure.
Since we have kept one of four identical copies of our na\"{\i}ve fermionic action we have essentially kept 4 of the 16 doubler modes.
This action can be interpreted as having a $2^4$ dimensional hyper cube unit cell with four flavors of fermions spread across it.
The four species of fermions that remain are called referred to as tastes.
While taste symmetry is exact in the continuum, it is broken by lattice artifacts.
Taste breaking can be especially problematic at strong coupling where the gauge fields are coarser.
Smearing terms can be added to the lattice action to mitigate or remove lattice artifacts associated with taste breaking.
I will discuss the smearing improvement that we use in our simulations in the next section.

In QCD studies, taste poses a unique problem when trying to study 2, or 2+1, or 2+1+1 flavors of fermions.
The 4 degenerate flavors have to be transformed into 1 or 2 flavors.
This is accomplished by a technique called rooting \cite{PhysRevD.72.114512}.
Although rooted calculations routinely reproduce correct results, there is still some controversy about the technique \cite{Creutz2007230,Creutz2007241,Kronfeld:2007ek}.
Fortunately in our studies we examine theories with multiples of 4 flavors thereby circumventing the rooting problem entirely.

The balance that staggered fermions strike makes them a good choice in many situations.
One benefit of staggered fermions is that they preserve a $U(1)_A$ chiral symmetry.
Wilson fermions explicitly break chiral symmetry while lattice actions that preserve chiral symmetry exactly are much more computationally expensive.
Finally, unlike Wilson fermions, there is a protected Goldstone mode which protects fermion masses from additive renormalization.

\subsection{Improved Fermions}

In the simulations detailed in chapters \ref{ch:MCRG}-\ref{ch:WMCRG} we use nHYP improved staggered fermions in our action.
nHYP smearing was introduced in \cite{Hasenfratz:2007rf} and reduces the problem of lattice artifacts, specifically taste breaking, in staggered actions \cite{Hasenfratz:2001tw}.

nHYP smearing is built out of three consecutive smearing steps that are restricted to the hypercubes associated with the link being smeared.
The first step:
\begin{align}
  \bar{\Gamma}_{n,\mu;\nu,\rho}&=(1-\alpha_3)U_{n,\mu}+\frac{\alpha_3}{2}\sum_{\pm\eta\neq\rho,\nu,\mu}U_{n,\eta}U_{n+\hat{\eta},\mu}U^\dagger_{n+\hat{\mu},\eta} \\
  \bar{V}&=\bar{\Gamma}(\bar{\Gamma}^\dagger\bar{\Gamma})^{-\frac{1}{2}},
\end{align}
where $\bar{V}=\Gamma(\Gamma^\dagger\Gamma)^\frac{1}{2}$ is a $U(N)$ projection.
The second step uses the links from the first smearing step and projection and performs another smearing and projection
\begin{align}
  \widetilde{\Gamma}_{n,\mu;\nu}&=(1-\alpha_2)U_{n,\mu}+\frac{\alpha_2}{4}\sum_{\pm\rho\neq\nu,\mu}\bar{V}_{n,\rho;\nu,\mu}\bar{V}_{n+\hat{\rho},\mu;\rho,\nu}\bar{V}^\dagger_{n+\hat{\mu},\rho;\nu,\mu} \\
  \widetilde{V}&=\widetilde{\Gamma}(\widetilde{\Gamma}^\dagger\widetilde{\Gamma})^{-\frac{1}{2}}.
\end{align}
Finally the last step is a third iteration of the smearing step
\begin{align}
  \Gamma_{n,\mu}&=(1-\alpha_1)U_{n,\mu}+\frac{\alpha_1}{6}\sum_{\nu\neq\mu}\widetilde{V}_{n,\nu;\mu}\widetilde{V}_{n+\hat{\nu},\mu;\nu}\widetilde{V}^\dagger_{n+\hat{\mu},\nu;\mu} \\
  V&=\Gamma(\Gamma^\dagger\Gamma)^{-\frac{1}{2}}.
\end{align}
Here the links $U_{n,\mu}$ are un-smeared and the nHYP smeared links are $V_{n,\mu}$.
The parameters $(\alpha_1,\alpha_2,\alpha_3)$ are parameters which can be tuned to control the amount of smearing.
The indices after the semi-colon are not included in the staple sum.
The link products
\begin{equation}
  \sqcap_{n,\mu}=\sum_{\nu\neq\mu}U_{n,\nu}U_{n+\hat{\nu},\mu}U^\dagger_{n+\hat{\mu},\nu}
\end{equation}
is the staple sum of the un-smeared link $U_{n,\mu}$.

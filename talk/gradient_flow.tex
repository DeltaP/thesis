                      %******************%
%********************% Gradient Flow      %********************%
                      %******************%

\section{Gradient (Wilson) Flow}
\addtocounter{framenumber}{-1}

\begin{frame}
  \frametitle{Gradient Flow - Origins}
  Use trivializing maps to improve the HMC algorithm used in configuration generation\newline
  Generate trivializing map by integrating a flow.
  \newline
  $U \xrightarrow{F^{-1}} V \xrightarrow{\text{HMC}} V' \xrightarrow{F} U'$
  \newline
  \begin{enumerate}
    \item rough links
    \item update links
    \item smooth links back
  \end{enumerate}
\end{frame}

\begin{frame}
  \frametitle{Wilson Flow - Definition}
    Wilson Action
    \begin{align*}
      S_w &= \frac{1}{g^2} \sum\limits_{p}\text{Re }\text{Tr}[\mathbb{I}-U(p)]
    \end{align*}
    Wilson Flow
    \begin{align*}
      V_t(x,\mu)|_{t=0} &= U(x,\mu) \\
      \dot{V}_t(x,\mu) &= -g_0^2[\nabla_{x,\mu}S_w(V_t)]V_t(x,\mu)
    \end{align*}

    The derivative is the natural $SU(3)$ valued differential operator with respect to $V_t(x,\mu)$.
\end{frame}

\begin{frame}
  \frametitle{Connection to Stout Smearing}
  \begin{center}
    \includegraphics[width=0.3\textwidth]{fig/stout.png}
  \end{center}
  \begin{columns}[T]
    \begin{column}{.5\textwidth}
      \begin{block}{}
        \begin{align*}
          Q = & \frac{i}{2}(\Sigma U^{\dagger}-U\Sigma^{\dagger}) \\
              & - \frac{i}{2N}\text{Tr}(\Sigma U^{\dagger}-U\Sigma^{\dagger})
        \end{align*}
        \begin{align*}
          W=e^{ipQ}U_{n,\mu}=(\mathbb{I}+iQ+...)U_{n,\mu}
        \end{align*}
      \end{block}
    \end{column}
    \begin{column}{.5\textwidth}
      \begin{block}{}
        \begin{itemize}
          \item $U$ is the link, $\Sigma$ is the staple sum
          \item Q is Hermitian
          \item Generates smeared links W that are in the group
        \end{itemize}
      \end{block}
    \end{column}
  \end{columns}
\end{frame}

\begin{frame}
  \frametitle{Stout Smearing as Wilson Flow}
  Integrating infinitesimal stout smearing steps is equivalent to finding the solution to the flow equations
  \begin{align*}
    U(x,\mu)\rightarrow U'(x,\mu)&=e^{\epsilon [Z(U)](x,\mu)}U(x,\mu) \\
    Z(U) &=-g^2_0\nabla_{x,\mu}S_w
  \end{align*}
\end{frame}

\begin{frame}
  \frametitle{Properties of the Gradient Flow}
  \begin{itemize}
    \item Forward flow smooths out fluctuations
    \item The flow does not change IR physics
    \item The flow is generated by infinitesimal stout smearing steps
  \end{itemize}
\end{frame}

\subsection{Application to MCRG}
\addtocounter{framenumber}{-1}

\begin{frame}
  \frametitle{Optimization with Gradient Flow}
  Wilson flow integrates infinitesimal smearing steps.
  \begin{columns}[T]
    \begin{column}{.4\textwidth}
      \begin{block}{}
        \includegraphics[width=\textwidth]{fig/wilson_flow_opt.png}
      \end{block}
    \end{column}
    \begin{column}{.6\textwidth}
      \begin{block}{}
        \begin{itemize}
          \item Wilson flow removes UV fluctuation
          \item Wilson flow does not change lattice spacing
          \item Moves system toward the RT
          \item Proceed with MCRG
        \end{itemize}
      \end{block}
    \end{column}
  \end{columns}
\end{frame}

\begin{frame}
  \frametitle{SU(3) $\text{N}_{\text{f}}=$ 12}
  \begin{columns}[c]
    \begin{column}{0.5\textwidth}
      \begin{itemize}
        \item fermion mass exactly zero
        \item APBC lattices
        \item fundamental adjoint gauge action
        \item nHYP smeared staggered fermions
      \end{itemize}
    \end{column}
    \begin{column}{0.5\textwidth}
      \begin{tabular}{| r | l |}\hline
        Volume & Bare Coupling ($\beta_F$) \\\hline
        $6^4$  & 3.4, 3.6, ... 7.8, 8.0 \\
        $8^4$  & 3.0, 3.2, ... 7.8, 8.0 \\
        $12^4$ & 3.4, 3.6, ... 7.8, 8.0 \\
        $16^4$ & 3.0, 3.2, ... 7.8, 8.0 \\
        $24^4$ & 4.0, 4.5, ... 7.5, 8.0 \\
        $32^4$ & 4.0, 5.0, ... 5.5, 6.0 \\
        $48^4$ & 6.0 \\ \hline
      \end{tabular}
    \end{column}
  \end{columns}
\end{frame}

\begin{frame}
  \frametitle{$6^4$ - $12^4$ - $24^4$ Matching}
  \begin{center}
    Errors indicate the spread in $s_b$ predicted by each observable.
  \end{center}
  \begin{columns}[T]
    \begin{column}{.3\textwidth}
      \begin{block}{}
        \begin{itemize}
          \item nHYP block transformation
          \item $n_b=3$
          \item $V_f=3^4$
        \end{itemize}
      \end{block}
    \end{column}
    \begin{column}{.7\textwidth}
      \begin{block}{}
        \includegraphics[width=\textwidth]{fig/12flav_6-12-24_3lm.pdf}
      \end{block}
    \end{column}
  \end{columns}
  \begin{center}
    \only<1>{
      Scheme 1:  \textcolor{red}{0.6 0.2 0.2}\quad\quad Scheme 2:  0.6 0.3 0.2 \\
      Scheme 3:  \textcolor{blue}{0.65 0.3 0.2}
    }
  \end{center}
\end{frame}

\begin{frame}
  \frametitle{$8^4$ - $16^4$ - $32^4$ Matching}
  \begin{center}
    Errors indicate the spread in $s_b$ predicted by each observable.
  \end{center}
  \begin{columns}[T]
    \begin{column}{.3\textwidth}
      \begin{block}{}
        \begin{itemize}
          \item nHYP block transformation
          \item $n_b=3$
          \item $V_f=4^4$
        \end{itemize}
      \end{block}
    \end{column}
    \begin{column}{.7\textwidth}
      \begin{block}{}
        \includegraphics[width=\textwidth]{fig/12flav_8-16-32_44_as_3lm.pdf}
      \end{block}
    \end{column}
  \end{columns}
  \begin{center}
    \only<1>{
      Scheme 1:  \textcolor{red}{0.6 0.2 0.2}\quad\quad Scheme 2:  0.6 0.3 0.2 \\
      Scheme 3:  \textcolor{blue}{0.65 0.3 0.2}
    }
    \only<2>{
      \begin{exampleblock}{}
        Now let us look at Scheme 1 and change $n_b$, $V_f$.
      \end{exampleblock}
    }
  \end{center}
\end{frame}

\begin{frame}
  \frametitle{$8^4$ - $16^4$ - $32^4$ Matching}
  \begin{columns}[T]
    \begin{column}{.3\textwidth}
      \begin{block}{}
        \begin{itemize}
          \item Scheme 1
          \item $n_b=3$ $V_f=3^4$
          \item \textcolor{blue}{$n_b=3$ $V_f=4^4$}
        \end{itemize}
      \end{block}
    \end{column}
    \begin{column}{.7\textwidth}
      \begin{block}{}
        \includegraphics[width=\textwidth]{fig/12flav_8-16-32_3lm_44.pdf}
      \end{block}
    \end{column}
  \end{columns}
\end{frame}

\begin{frame}
  \frametitle{$8^4$ - $16^4$ - $32^4$ Matching}
  \begin{columns}[T]
    \begin{column}{.3\textwidth}
      \begin{block}{}
        \begin{itemize}
          \item Scheme 1
          \item $n_b=3$ $V_f=3^4$
          \item \textcolor{blue}{$n_b=3$ $V_f=4^4$}
          \item \textcolor{red}{$n_b=4$ $V_f=2^4$}
        \end{itemize}
      \end{block}
    \end{column}
    \begin{column}{.7\textwidth}
      \begin{block}{}
        \includegraphics[width=\textwidth]{fig/12flav_8-16-32_3lm_22.pdf}
      \end{block}
    \end{column}
  \end{columns}
\end{frame}

\begin{frame}
  \frametitle{$12^4$ - $24^4$ - $48^4$ Matching}
  \begin{columns}[T]
    \begin{column}{.3\textwidth}
      \begin{block}{}
        \begin{itemize}
          \item Scheme 1
          \item \textcolor{red}{$n_b=3$ $V_f=3^4$}
          \item $n_b=3$ $V_f=4^4$
          \item \textcolor{blue}{$n_b=4$ $V_f=2^4$}
          \item \textcolor{magenta}{$n_b=3$ $V_f=6^4$}
          \item \textcolor{cyan}{$n_b=4$ $V_f=3^4$}
        \end{itemize}
      \end{block}
    \end{column}
    \begin{column}{.7\textwidth}
      \begin{block}{}
        \includegraphics[width=\textwidth]{fig/12flav_12-24-48_3lm.pdf}
      \end{block}
    \end{column}
  \end{columns}
\end{frame}

\begin{frame}
  \frametitle{Summary - WMCRG}
  \begin{itemize}
    \item New work on an established method
    \item With Wilson flow MCRG we can find a unique step scaling function
    \item Wilson flow MCRG predicts an IRFP for SU(3) $N_f=12$ in all analysis
    \item Wilson flow MCRG is computationally inexpensive; does not rely on perturbation theory
  \end{itemize}
\end{frame}

\subsection{Wilson Flow Step Scaling Function}
\addtocounter{framenumber}{-1}

\begin{frame}
  \frametitle{Another Approach}
  New RG approaches have been proposed that use Wilson Flow to find the renormalized step scaling function.
  \begin{itemize}
    \item $g_c^2(\mu)$
    \item $\mu = \frac{1}{\sqrt{8t}}$
    \item $t = \frac{(cL)^2}{8} \rightarrow \mu = \frac{1}{cL}$ scale is set by the volume
    \item $c\approx 0.2-0.3$
    \item From gradient flow define renormalized coupling $g_c^2=\frac{128\pi^2\left<t^2E(t)\right>}{3(N^2-1)(1+\delta_c)}$
    \item $\beta_{lat}(g^2_c; s) = \frac{g^2_c(L) - g^2_c(sL)}{\log(s^2)}$
  \end{itemize}
\end{frame}

\begin{frame}
  \frametitle{First Look}
  Very different from asymptotic freedom.
  \begin{columns}
    \begin{column}{0.65\textwidth}
      \includegraphics[width=1.2\textwidth]{fig/wflow.png}
    \end{column}
    \begin{column}{0.35\textwidth}
      \begin{center}
        $c=0.3$
        \vspace{24pt}
        fixed $\beta$ in SC:\\
        $g^2(L) \downarrow$ as $L \uparrow$\\
        \vspace{24pt}
        crossings = IRFP
      \end{center}
    \end{column}
  \end{columns}
\end{frame}

\begin{frame}
  \frametitle{Continuum Extrapolation}
  This is not $\beta = \infty$ continuum limit.
  \begin{columns}
    \begin{column}{0.7\textwidth}
      \includegraphics[width=1.2\textwidth]{fig/wflow_continuum.png}
    \end{column}
    \begin{column}{0.3\textwidth}
      \begin{center}
        fix $g^2(L)=g_0^2$\\\vspace{24pt}
        compare\\$g^2(2L)-g^2(L)$\\\vspace{24pt}
        $L\rightarrow \infty$\\requires\\decreasing $\beta$
      \end{center}
    \end{column}
  \end{columns}
\end{frame}

\begin{frame}
  \frametitle{Focus on the Fixed Point}
  fit\footnote{Fodor, Zoltan et al. JHEP 1211 (2012) 007} $$\frac{\beta_F}{6}-\frac{1}{g^2(\beta_F)}=\sum_{m=0}^3 c_m(\frac{6}{\beta_F})^m$$
    locate intersection
  \includegraphics[width=0.5\textwidth]{fig/1212-2424_wflow.png}\includegraphics[width=0.5\textwidth]{fig/1616-3232_wflow.png}
  \newline
  \emph{Does the fixed point survive the continuum limit?}
\end{frame}

\begin{frame}
  \frametitle{Fixed Point Continuum Extrapolation}
  \begin{columns}[T]
    \begin{column}{.65\textwidth}
      \includegraphics[width=\textwidth]{fig/cont_extrap_no_improv.png}
    \end{column}
    \begin{column}{.35\textwidth}
      \begin{equation*}
        \lim_{(\frac{a}{L})^2\to 0} g^2_*(L;s)\equiv g^2_*
      \end{equation*}
      \begin{equation*}
        g^2_*=5.50\pm0.12
      \end{equation*}
      \footnotesize
      Black line: continuum extrapolation\\
      \vspace{12pt}
      Black point: continuum limit\\
      \vspace{12pt}
      Errors come from uncertainty in the fits at the crossing
    \end{column}
  \end{columns}
  \vspace{12pt}
  \emph{Can we improve the continuum extrapolation?}
\end{frame}

\begin{frame}
  \frametitle{Wilson Flow Optimization}
  We want to remove cutoff effects without using $c>0.3$ which sacrifices statical uncertainty.\\
  \vspace{12pt}
  Replace:
  \begin{equation*}
    g_c^2=\frac{128\pi^2\left<t^2E(t)\right>}{3(N^2-1)(1+\delta_c)}
  \end{equation*}
  with
  \begin{equation*}
    \widetilde{g}_c^2=g_c^2\frac{\left<E(t+\tau_0)\right>}{\left<E(t)\right>}
  \end{equation*}
  keeping $\tau_0\ll t$\\
  \vspace{12pt}
  \emph{No additional computation is necessary!}
\end{frame}

\begin{frame}
  \frametitle{Combined Fit - Vary $\tau_0$}
  \begin{columns}[T]
    \begin{column}{.7\textwidth}
      \includegraphics[width=\textwidth]{fig/fit_s2_c02.pdf}
    \end{column}
    \begin{column}{.3\textwidth}
      \begin{itemize}
        \item fix $c=0.2$
        \item fix $s=2$
        \item vary $\tau_0$
      \end{itemize}
      \vspace{24pt}
      $g^2_*=6.21\pm0.25$
    \end{column}
  \end{columns}
  \vspace{12pt}
  Dashed lines are linear fits with a common intercept\\
  \begin{center}\emph{The optimal $\tau_0\approx0.04$}\end{center}
\end{frame}

\begin{frame}
  \frametitle{Combined Fit - $\tau_{opt}$}
  \begin{columns}[T]
    \begin{column}{.7\textwidth}
      \includegraphics[width=\textwidth]{fig/fit_c02_tau004.pdf}
    \end{column}
    \begin{column}{.3\textwidth}
      \begin{itemize}
        \item fix $c=0.2$
        \item fix $\tau_0=0.04$
        \item vary $s$
      \end{itemize}
      \vspace{24pt}
      $g^2_*=6.18\pm0.20$
    \end{column}
  \end{columns}
  \vspace{12pt}
  Dashed lines are linear fits with a common intercept\\
  Extrapolation has negligible $\mathcal{O}(a^2)$ dependence
\end{frame}

\begin{frame}
  \frametitle{Summary - Optimized Gradient Flow Step Scaling}
  \begin{itemize}
    \item We calculate the gradient flow for $SU(3)$ $N_f=12$ and see an IRFP
    \item The IRFP survives the continuum limit
    \item We introduce an improvement that removes $\mathcal{O}(a^2)$ dependence
from the continuum extrapolation
  \end{itemize}
\end{frame}

\begin{frame}
  \frametitle{Conclusions}
  \begin{itemize}
    \item We have calculated the bare step scaling function and renormalized step scaling function for $SU(3)$ $N_f=12$ using 3 methods
    \item All calculations indicate that $SU(3)$ $N_f=12$ has in infrared fixed point
    \item These step scaling calculations also corroborate our groups finite temperature and eigenvalue studies
    \item These improved methods are computationally inexpensive and will be useful to further lattice calculations
  \end{itemize}
\end{frame}

\begin{frame}
  \frametitle{Acknowledgments}
  \begin{itemize}
    \item{Anna, David, Anqi, Yuzhi}
    \item{DOE Office of Science Graduate Student Fellowship}
    \item{USQCD}
    \item{University of Colorado Research Computing (Janus)}
    \item{NSF XSEDE}
    \item{Thank you for watching}
  \end{itemize}
\end{frame}

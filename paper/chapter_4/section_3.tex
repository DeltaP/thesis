% 4.3
% 8 and 12 Flavor Results

In this section I employ the two lattice matching technique described earlier in this chapter to study SU(3) gauge theories with 8 and 12 chiral fermions in the fundamental representation.
We proceed by repeatedly applying an nHYP RG blocking transformations with scale factor $s = 2$.
The lattices used in this study have volumes $24^3\X48$, $12^3\X24$ and $6^3\X12$.
On the largest $24^3\X48$ lattices that we use in this current study, we work with fermion masses $m = 0.0025$ to stay near the $m = 0$ critical surface.
Under RG blocking with scale factor $s$, the fermion mass changes as $s^{1 + \ga_m}$ where $\ga_m$ is the mass anomalous dimension.
Therefore we use $m = 0.01$ on $12^3\X24$ and $m = 0.02$ on $6^3\X12$ lattices.
We have explicitly checked that these masses are small enough to introduce only negligible finite-mass effects, by generating lattices with $m = 0$ for some points and obtaining indistinguishable results.

Under RG blocking on the $m = 0$ critical surface, the system flows toward the renormalized trajectory in irrelevant directions, and along it in the relevant direction.
By blocking the larger lattices (with $\be_F$) $n_b$ times and the smaller lattices (with $\be_F'$) only $n_b - 1$ times, we obtain blocked systems with the same lattice volume.
If these blocked systems have both flowed to the same point on the renormalized trajectory, then we can conclude that $\xi(\be_F) = 2\xi(\be_F')$ on the unblocked systems, as desired.

We determine whether the blocked systems have flowed to the same point on the renormalized trajectory by matching several short-range gauge observables: the plaquette, all three six-link loops, and two planar eight-link loops.
For a given $\be_F$, each observable may predict a different $\De\be_F \equiv \be_F - \be_F'$.
The spread in these results is a systematic error that dominates our uncertainties.
In principal more observables can be added, our choices are limited by the final volume of the blocked lattice.

In an IR-conformal system, the gauge coupling that is relevant at the perturbative gaussian FP becomes irrelevant at the IRFP.
The renormalized trajectory connects these two fixed points.
When RG flows approach this renormalized trajectory, the two-lattice matching can be performed and interpreted the same way as in confining systems.
In this region the gauge coupling flows to stronger couplings, $\De\be_F > 0$ corresponding to a negative RG \be function.
The situation is less clear at stronger couplings where we might na\"ively expect backward flow.
If there is no ultraviolet FP in this region to drive the RG flow along a renormalized trajectory, the two-lattice matching might become meaningless.
This issue affects every method that attempts to determine the flow of the gauge coupling in IR-conformal systems at strong coupling.
In all published studies that report an IRFP, backward flow has only been observed in a very limited range of couplings in the immediate vicinity of the IRFP (cf.~\refcite{Giedt:2012LAT}).

Since we can block our lattices only a few times, we must optimize the two-lattice matching by requiring that consecutive RG blocking steps yield the same $\De\be_F$.
We identify the optimized $\De\be_F$ with the bare step-scaling function $s_b$.
The traditional technique reported in this chapter optimizes the RG blocking transformation (renormalization scheme).
The new method we propose in \secref{ch:WMCRG} instead applies the Wilson flow to the lattice system prior to RG blocking, and optimizes the flow time $t_f$.

As in Refs.~\cite{Hasenfratz:2011xn, Hasenfratz:2011np}, we use RG blocking transformations that include two sequential HYP smearings with parameters $(\al, 0.2, 0.2)$, and optimize \al as shown in the left panel of \fig{fig:opt}.
Qualitatively, this optimization finds the renormalization scheme for which the renormalized trajectory passes as close as possible to the lattice system with coupling $\be_F$.
Without optimization, residual flows in irrelevant directions can distort the results: this is the reason $\De\be_F$ changes with \al in \fig{fig:opt}, and also explains why increasing the number of blocking steps reduces this $\al$-dependence.

\begin{figure}[htpb]
  \centering
  \includegraphics[height=3in]{\cnum{4}/fig/plots/optimization.pdf}
  \caption{Optimization of the HYP-smearing parameter \al in the RG blocking transformation, for $\be_F = 5.0$.  The uncertainties on the data points are dominated by averaging over the different observables as described in the text.}
  \label{fig:opt}
\end{figure}

We note that finite-volume effects are minimized by carrying out the optimization on blocked lattices of the same volume, which was reported by \refcite{Hasenfratz:2011xn}.
That is, we should compare $\De\be_F$ from matching $12^3\X24$ blocked to $3^3\X6$ vs.\ $6^3\X12$ blocked to $3^3\X6$ with that from matching $24^3\X48$ blocked to $3^3\X6$ vs.\ $12^3\X24$ blocked to $3^3\X6$.
As in Refs.~\cite{Hasenfratz:2011xn, Hasenfratz:2011np}, we do not explore weak enough couplings to recover the two-loop perturbative predictions $s_b \approx 0.3$ for $N_f = 12$.

\subsection{8 Flavors}
In our study of SU(3) gauge theories with 8 flavors we calculated the step scaling function at four couplings corresponding to $\beta_F = 5.4, 5.6, 6, 7$.
We were unable to perform matching at stronger coupling due to the appearance of the \Sb phase at $\beta_f=4.6$.
At weaker couplings our results for the step scaling function are lower than those predicted by two loop perturbation theory.
Perturbation theory predicts a value of $s_b \approx 0.6$.
This implies that our simulations are at couplings too strong to perform comparison to perturbation theory.
Figure \ref{fig:8MCRG} displays our final results.
The error bars are indicative of the spread the step scaling function predicted by different observables.

It is clear that the step scaling function is far removed from zero.
This indicates that the 8 flavor theory does not have an infrared fixed point.
While the calculated step scaling function is not in explicit agreement with perturbative value it is somewhat close.
Furthermore interpolating between the our measured $s_b$ and the perturbative value does not require a fixed point to develop.
This picture is consistent with 8 flavors being chirally broken and confining.

\begin{figure}[htpb]
  \centering
  \includegraphics[height=3in]{\cnum{4}/fig/plots/8flavor.pdf}
  \caption{Results for the bare step-scaling function $s_b$ from traditional MCRG two-lattice matching with $24^3\X48$, $12^3\X24$ and $6^3\X12$ lattice volumes for $N_f = 8$.  The blue dashed lines are perturbative predictions for asymptotically weak coupling.}
  \label{fig:8MCRG}
\end{figure}

\subsection{12 Flavors}
Our investigation of SU(3) gauge theory with 12 flavors reveals behavior very different from the 8 flavor results.
We calculated the step scaling function at $\beta_F=5,6,7,8,9,10$.
As with the 8 flavor case we were unable to perform matching at stronger couplings because of the \Sb phase at $\beta=2.7$.
At the weakest couplings studied, we find that our results for the step scaling function are smaller than those predicted by the two loop perturbative calculation which predicts $s_b\approx 0.3$.
As with the 8 flavor results, this implies that we are at strong enough coupling that perturbation theory is unreliable.

The results shown in \ref{fig:12MCRG} show that the step scaling function for $\beta_F=8,9,10$ is positive while the step scaling function at $\beta_F=5,6,7$ is negative or close to zero.
This is consistent with backwards flow at strong couping and an infrared fixed point at $\beta_F<8$.
Although we are unable to cleanly identify where the fixed point is due to reasons I will elucidate in the next section, the results are not consistent with a chirally broken theory.
\begin{figure}[htpb]
  \centering
  \includegraphics[height=3in]{\cnum{4}/fig/plots/12flavor.pdf}
  \caption{Results for the bare step-scaling function $s_b$ from traditional MCRG two-lattice matching with $24^3\X48$, $12^3\X24$ and $6^3\X12$ lattice volumes for $N_f = 12$.  The blue dashed lines are perturbative predictions for asymptotically weak coupling.}
  \label{fig:12MCRG}
\end{figure}

\subsection{The Case of the Wandering Fixed Point}
Recall that our optimization of the RG blocking transformation means that we use a different renormalization scheme for each coupling $\be_F$, so these bare step scaling functions are composites of several different discrete \be functions.
In an IR conformal system, each renormalization scheme has a fixed point because the existence of the fixed point is scheme independent.
The location of the fixed point however is a scheme dependant property.
Therefore it is reasonable to expect the IRFP to be located at a different point in coupling space for each $\beta_F$ that we calculate $s_b$ for.
For example, with $N_f = 12$ at $5 \leq \be_F \leq 6$, our optimization selects renormalization schemes with the fixed point near $\be_F$, so that $s_b$ is roughly consistent with zero over an extended range.

\begin{figure}[htpb]
  \centering
  \includegraphics[height=3in]{\cnum{4}/fig/opt_prob.png}
  \caption{An illustration of how optimizing the block transformation can result in difficulties locating an IRFP, $\beta$ is the relevant gauge coupling and $\beta'$ are irrelevant couplings.  The upper figure shows the renormalized trajectory in red, green, and blue found by optimizing the RG transformation at $\beta_1, \beta_2, and \beta_3$ respectively.  The location of the IRFP changes in each renormalized trajectory, in this picture the IRFP is moved to the coupling we perform MCRG at.  The resulting $s_b$ is consistent with zero across a wide range of couplings. }
  \label{fig:opt_prob}
\end{figure}

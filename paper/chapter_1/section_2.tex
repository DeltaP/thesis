% 1.2
% Organization

This thesis has two primary objectives.
First I want to demonstrate new techniques that can be used to calculate the beta function.
These techniques are applicable to any lattice calculation and therefore will be of use to any lattice practitioner.
The second goal is to locate the conformal window of strongly coupled gauge theories with $N_c$ colours and $N_f$ fermions.
Naturally I will use the techniques I develop to probe the conformal window.

Before delving into my own work, I use chapter 2 to set the stage.
I begin the section with a brief overview of the standard model, the Higgs mechanism, and why our current understanding of the universe is note complete.
I then discuss the idea of a composite Higgs and why the lattice is needed to study composite Higgs theories.

Chapter 3 provides the reader with a brief overview of lattice gauge theory.  
This topic is a subject of many books, and my goal is not to cover this subject comprehensively.
Rather I want to provide the reader with a flavor of how a lattice calculation is performed and highlight areas that are useful in later sections.

Chapter 4 - 6 document the body of work that is this thesis.
Chapter 4 concerns my early work using Monte Carlo Renormalization Group (MCRG) on SU(3) $N_f$ = 8 and SU(3) $N_f$ = 12 theories.
In chapter 5 I introduce the gradient flow and how it can be used to calculate the renormalized beta function.
I also use this chapter to discuss an optimization improvement that our group developed.
Results for for SU(3) $N_f$ = 4 and SU(3) $N_f$ = 12 are given.
Finally chapter 6 shows how the Wilson Flow can be used as an optimization step for MCRG.
Results from SU(3) $N_f$ = 12 are given.

In my concluding chapter I make some final remarks about interpreting our results.
I will also summarize results from other methods that our group used to study this system.
As expected all of our results show that SU(3) gauge theories with 12 chiral fermions are conformal.

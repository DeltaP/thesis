% 6.3
% 12 Flavor Results

\begin{figure}[ht]
  \centering
  \includegraphics[height=3 in]{\cnum{6}/fig/12flav_6-12-24_3lm}
  \caption{The bare step-scaling function $s_b$ predicted by three-lattice matching with $6^4$, $12^4$ and $24^4$ lattices blocked down to $3^4$, comparing three different renormalization schemes.  The error bars come from the standard deviation of predictions using the different observables discussed in \protect\secref{sec:mcrg}.}
  \label{fig:24to3}
\end{figure}

\begin{figure}[ht]
  \centering
  \includegraphics[height=3 in]{\cnum{6}/fig/12flav_8-16-32_44_as_3lm}
  \caption{As in \protect\fig{fig:24to3}, the bare step-scaling function $s_b$ for three different renormalization schemes from three-lattice matching, now using $8^4$, $16^4$ and $32^4$ lattices blocked down to $4^4$.}
  \label{fig:32to4}
\end{figure}

Our WMCRG results for the 12-flavor system are obtained on gauge configurations generated with exactly massless fermions.
Our lattice action uses nHYP-smeared staggered fermions as described in \refcite{Cheng:2011ic}, and to run with $m = 0$ we employ anti-periodic boundary conditions in all four directions.
All of our analyses are carried out at couplings weak enough to avoid the unusual strong-coupling ``$\Sb$'' phase discussed by Refs.~\cite{Cheng:2011ic, Hasenfratz:2013uha}.

We perform three-lattice matching with volumes $6^4$--$12^4$--$24^4$ and $8^4$--$16^4$--$32^4$.
Three-lattice matching is based on two sequential two-lattice matching steps, to minimize finite-volume effects~\cite{Hasenfratz:2011xn}.
Both two-lattice matching steps are carried out on the same final volume $V_f$.
We denote the number of blocking steps on the largest volume by $n_b$, and tune the length of the initial Wilson flow by requiring that the last two blocking steps predict the same step-scaling function.
Using the $8^4$--$16^4$--$32^4$ data we determine the bare step-scaling function for $n_b = 3$ and $V_f = 4^4$ as well as $n_b = 4$ and $V_f = 2^4$, while the $6^4$--$12^4$--$24^4$ data set is blocked to a final volume $V_f = 3^4$ ($n_b = 3$).
This allows us to explore the effects of both the final volume and the number of blocking steps.
We investigate three renormalization schemes by changing the HYP smearing parameters in our blocking transformation~\cite{Petropoulos:2012mg}: scheme~1 uses smearing parameters (0.6, 0.2, 0.2), scheme~2 uses (0.6, 0.3, 0.2) and scheme~3 uses (0.65, 0.3, 0.2).

Figs.~\ref{fig:24to3}, \ref{fig:32to4} and \ref{fig:scheme1} present representative results for 12 flavors.
All of the bare step-scaling functions clearly show $s_b = 0$, signalling an infrared fixed point, for every $n_b$, $V_f$ and renormalization scheme.
Appropriately for an IR-conformal system, the location of the fixed point is scheme dependent.
We observe that the fixed point moves to stronger coupling as the HYP smearing parameters in the RG blocking transformation increase.

When we block our $8^4$, $16^4$ and $32^4$ lattices down to a final volume $V_f = 2^4$ (corresponding to $n_b = 4$), the observables become very noisy, making matching more difficult.
The problem grows worse as the HYP smearing parameters increase, and our current statistics do not allow reliable three-lattice matching for $V_f = 2^4$ in schemes~2 and 3.
To resolve this issue, we are accumulating more statistics in existing $32^4$ runs, and generating additional $32^4$ ensembles at more values of the gauge coupling $\be_F$.
These additional data will also improve our results for scheme~1, which we show in \fig{fig:scheme1}.
Different volumes and $n_b$ do not produce identical results in scheme~1, suggesting that the corresponding systematic effects are still non-negligible.
We can estimate finite-volume effects by comparing $n_b = 3$ with $V_f = 3^4$ and $V_f = 4^4$.
Systematic effects due to $n_b$ can be estimated from $n_b = 4$ and $V_f = 2^4$, but this is difficult due to the noise in the $2^4$ data.
Even treating the spread in the results shown in \fig{fig:scheme1} as a systematic uncertainty, we still obtain a clear zero in the bare step-scaling function, indicating an IR fixed point.

\begin{figure}[th]
  \centering
  \includegraphics[height=3 in]{\cnum{6}/fig/12flav_8-16-32_3lm_22}
  \caption{The bare step-scaling function $s_b$ for scheme~1, comparing three-lattice matching using different volumes: $6^4$, $12^4$ and $24^4$ lattices blocked down to $3^4$ (black $\times$s) as well as $8^4$, $16^4$ and $32^4$ lattices blocked down to $4^4$ (blue bursts) and $2^4$ (red crosses).}
  \label{fig:scheme1}
\end{figure}

In this chapter we have shown how the Wilson-flow-optimized MCRG two-lattice matching procedure proposed in \refcite{Petropoulos:2012mg} improves upon traditional lattice renormalization group techniques.
By optimizing the flow time for a fixed RG blocking transformation, WMCRG predicts a bare step-scaling function $s_b$ that corresponds to a unique discrete \be function.
Applying WMCRG to new 12-flavor ensembles generated with exactly massless fermions, we observe an infrared fixed point in $s_b$.
The fixed point is present for all investigated lattice volumes, number of blocking steps and renormalization schemes, even after accounting for systematic effects indicated by \fig{fig:scheme1}.
This result reinforces the IR-conformal interpretation of our complementary $N_f = 12$ studies of phase transitions~\cite{Schaich:2012fr, Hasenfratz:2013uha}, the Dirac eigenmode number~\cite{Cheng:2013eu, Cheng:2013bca}, and finite-size scaling~\cite{Hasenfratz:2013eka}.

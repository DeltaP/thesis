% 2.3
% Beyond the Standard Model

While the Standard Model is the pinnacle of our current understanding of the universe it is not the last word.
There are many phenomena that the standard model does not explain.
Furthermore there are theoretically troubling aspects to the standard model that leaves more to be desired.

Perhaps the biggest omission in the standard model is a quantum field theory of Gravity.
Our current understanding of gravity is Einstein's theory of General Relativity which is another thoroughly verified theory.
General Relativity is incompatible with quantum mechanics, and to date a force carrying boson, the graviton, has eluded detection.
A related problem is that gravity is so much weaker than the other three fundamental forces.
Another unexplained phenomena is neutrino oscillations.
In nature we have observed the flavor of neutrinos to change.
These oscillations prove that the neutrino's are not massless particles as the standard model previously had us believe.
Finally we currently understand from astrophysical observations that the standard model only accounts for 4\% of the energy budget of the universe.
Dark matter, which has eluded detection composes of another 20\% of the energy budget.
Dark Energy, which is responsible for the acceleration of the expansion of the universe composes 75\% of our energy budget.

In addition to not explaining the entire range of physical phenomena that we observe (or fail to observe) the standard model is in a way that suggests that 

The hierarchy problem 

Strong CP problem

These issues have pushed researchers to look for extensions or alternatives to the standard model that answer one or more of these unresolved questions.
One area of active research has been the hierarchy problem.
The state of affairs for almost 40 years was that we knew there had to be a Higgs mechanism.
The Higgs Boson had not yet been discovered and therefore it was possible to build models that solve the hierarchy problem by introducing new physics at the electroweak scale often in the form of new fields or new symmetries.
Super Symmetry is one such approach that postulates a symmetry between fermions and bosons.
Another approach which I will describe further in the next section is for the Higgs particle to be a composite particle.
A consequence of all of these theories is new particles.
Now we know the value of the Higgs mass and no other particles have been discovered by the LHC yet, ruling out the simplest incarnations of these models.

% 1.1
% Motivation

The standard model (SM) of particle physics is among the most rigorously tested theories describing the physical world we live in.
It describes three of the four fundamental forces of nature in terms of quantum interactions of fundamental particles.
Recently, on July 4, 2012, the discovery of the Higgs particle was announced.
This is an exciting discovery for many reasons.
First of all, the Higgs is predicted by the standard model as are many of its properties.
In this sense its discovery is a good test of the standard model.
Additionally, the Higgs was the last `missing' piece of the standard model, now we have a complete theory.
Finally, the discovery of the Higgs puts very hard constraints on theories of Beyond Standard Model (BSM) physics.

The Higgs particle in the standard model is an excitation of the Higgs field.
The Higgs field is responsible for electroweak symmetry breaking (EWSB).
EWSB explains how the $W^{\pm}$ and $Z$ bosons acquire a mass that would otherwise be forbidden.
Explicitly putting a mass term in the Electroweak theory violates local gauge invariance, a critical component of the theory.
By allowing the $W^{\pm}$ and $Z$ bosons to couple to the Higgs field, they are allowed to become massive without violating local gauge invariance.
Additionally, in the EWSB process, the Higgs develops a vacuum expectation value (VEV).
By coupling to the Higgs VEV, the standard model fermions acquire mass as well.
In this way the Higgs provides an explanation of how all the fundamental particles gain their mass.

The Higgs field in the standard model is a scalar field.
This leads to the hierarchy problem (also equivalent to the naturalness or fine tuning problem) as the Higgs mass will be dependant on the UV cutoff of the standard model.
If the standard model is the complete story and there is no new physics just above the electroweak scale then the SM cutoff will be at a very high scale.
Naïvely, we expect the Higgs mass to be at the same order as the cutoff.
The fact the Higgs mass is 126 GeV requires fine tuning in the electroweak sector to cancel the UV divergences.

There have been many solutions proposed as an alternative to the Standard Model that don't require fine tuning.
Supersymmetry (SUSY) is one such proposal that achieves cancellations by introducing a new symmetry between bosonic and fermionic fields.
Since the bosons and fermions enter loop calculations with opposite signs there is no need for fine tuning.
Most models of SUSY predict a spectrum of super partner masses near the Higgs mass.
The fact that we have not yet observed any particles besides the Higgs puts constraints on SUSY models.
Additionally most versions of SUSY that have not been ruled out replace the standard model hierarchy problem with a new  `little' hierarchy problem where fine tuning is still required but it is not as severe as it is in the standard model.

Another natural possibility is that EWSB is the result of new strong dynamics at the TeV scale.
Under this prescription, EWSB happens much like spontaneous chiral symmetry breaking in quantum chromodynamics (QCD).
QCD is the only strongly coupled sector of the standard model and is responsible for the strong nuclear force.
Theories of EWSB that rely on strong dynamics are generally called Technicolor models.
Technicolor is also highly constrained by the Higgs mass, the top quark mass, and precision electroweak measurements that show flavor changing neutral currents are highly suppressed.

It is natural to draw insights from the parallels of Technicolor and QCD or directly from perturbation theory, however such insights do not paint an accurate picture.
The strongly coupled nature of these theories demands that they are studied through fully nonperturbative means.
Therefore, lattice gauge theory is the only controlled way to study these theories.
In lattice gauge theory we discretized euclidean space time into a regular grid of sites connected by links.
The Lagrangian describing the theory is also discretized in such a way that in the continuum limit the original theory is recovered.
The continuum limit is the limit that spacing between lattice sites is infinitesimally small.
Lattice QCD, the study of QCD on the lattice, has been extremely successfully in understanding QCD.
Many nonperturbative effects such as confinement, hadronization, chiral symmetry breaking, and instantons have been studied extensively.
Many of these properties are completely inaccessible through analytic techniques while others could not be fully understood.

The ultimate goal of efforts in this field is to find a theory can explain electroweak symmetry breaking, as it is observed in nature, through new strong dynamics.
Finding such a theory is highly nontrivial and a great deal of theoretical and computational effort has gone into looking for such a theory.
The lattice community has set out to do this in a controlled and systematic way: map out the space of all possible theories and see what we find.
To date a lot of the focus of the lattice community has been on locating the conformal window and developing techniques to distinguish conformal from chiraly broken theories.
The conformal window describes a class of theories that develop an attractive infrared fixed point.
These theories are scale invariant and are not candidates for Technicolor.

This thesis has two primary objectives.
The first goal is a physics goal:  map out the behavior of $SU(3)$ gauge theories with $N_f$ fermions in the fundamental representation.
For technical reasons we choose to study theories with 4, 8, 12, and 16 flavors.
4 and 16 flavor theories are already well understood so the focus of our work is on 8 and 12 flavor systems.

My second objective is to develop tools that make the first goal possible.
Towards this end I introduce improvements to lattice step scaling calculation techniques.
I explore how these techniques are different and how to implement them in a calculation.
Finally these techniques are applicable to any lattice calculation and therefore will be of use to any lattice practitioner.

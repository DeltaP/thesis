\chapter{Conclusion}
\label{ch:conclusion}

Truly it is an exciting time in particle physics.
For almost two decades since the discovery of the top quark no new fundamental particles were discovered in collider experiments.
After a long wait the Higgs particle has been discovered, completing the standard model.
Although the standard model is complete we know it must be an effective theory and our knowledge of the Higgs mass puts constraints on beyond standard model physics.

Many proposals for beyond standard model physics, including technicolor, are strongly coupled theories and thus inherently nonperturbative.
Since the lattice offers the only controlled means of studying non perturbative field theories in a controlled manner, it is natural that strongly coupled beyond standard model is an active area of lattice research.
Studying BSM physics on the lattices has created many new challenges for the lattice community to solve.
Unlike QCD, we don't know the answer to most questions before hand, and there are no experimental results to compare lattice results with.
Furthermore just because a technique is successful when studying QCD does not mean that it will work just as well in systems that are very different from QCD.
Accordingly it is important to approach each theory studied carefully and with an open mind.
Only when several methods converge on the same result can that result be trusted.
Additionally since there is more than one way to put a continuum theory on the lattice, it is important to understand the effect of the lattice action and lattice artifacts.

The lattice search for viable technicolor theories has focused on exploring gauge theories with $SU(N_c)$ colors and $N_f$ fermions in some representation $R$.
By changing these parameters generates theories with dramatically different behavior.
Theories with a small number of fermionic flavors in a lower representation behave similarly to QCD.
If more fermionic degrees of freedom are added the theory develops and infrared fixed point.
If enough degrees of freedom are added the theory will loose asymptotic freedom.
The location of the conformal window in the parameter space $N_c, N_f,$ and $R$ is fundamentally a question of strong dynamics.
Perturbative and quasi pertrubative calculations exist for the bounds of the conformal window but they can only serve as a guide to locate interesting theories for numerical studies.

Ultimately we are interested in the behavior of theories that may exhibit a slowly running coupling.
This behavior, also called walking, may exist just below the conformal window.
In a walking theory $\gamma_m$ should be of order 1.
Such a theory is strongly favored by limits on flavor changing neutral currents in the SM.
To date a viable walking theory has not yet been found.

Many groups, including our own have explored the conformal window with several goals in mind.
First understand the extent of the conformal window.
Second improve lattice techniques in conformal systems.
Third explore the bottom of the conformal window for walking behavior.

In this thesis I have discussed three methods that can be used to understand the $\beta$ function.
These methods are general and work for confining and conformal systems, they have the benefit that distinguishing between the two is straight forward.
I first discussed MCRG and presented results for $SU(3)$ gauge theory with $N_f=8$ and $N_f=12$.
Our results were consistent with the 12 flavor theory exhibiting an IRFP.
The 8 flavor theory did not show a fixed point and appeared to be chirally broken and confining.

Next I introduced the gradient flow step scaling and showed results for $SU(3)$ gauge theory with $N_f=4$ and $N_f$=12.
The 4 flavor results were simply to show that our improvement works in a theory we know to be chirally broken.
They also serve as a contrast to the 12 flavor results which clearly indicate an IRFP.
Another group has recently applied this technique, with our improvement, to $SU(3)$ gauge theory with 8 flavors \cite{} their results show that while the theory runs slowly they can not locate an IRFP.

Finally I introduced an improvement to MCRG that uses the Wilson flow as an optimization.
We call this technique Wilson Flow MCRG or WMCRG.
WMCRG is like MCRG in that we are calculating the discrete step scaling function.
By using the Wilson Flow as an optimization we are able to probe a single renormalized trajectory.
Results from WMCRG for $SU(3)$ gauge theory of 12 flavors of fermions in the fundamental representation shows very clear evidence of a fixed point.

One of the benefits of these three techniques is that they are computationally inexpensive and do not require specific lattice dimensions or special boundary conditions.
That is the lattices that we use for these step scaling studies are able to be used in other studies as well.
Our group has been involved with several other analysis.

Finite temperature phase transitions

eigenmodes

fininte size scaling

For several years the IR physics of $SU(3)$ gauge theory with 12 flavors of fermions in the fundamental representation was a topic of debate.
Several early papers \cite{} 
A consensus has emerged that the theory is indeed conformal.
This consensus is a result of improvements in our understanding of quantum field theories as well as implements in lattice techniques used to study those quantum field theories.

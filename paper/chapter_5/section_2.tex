% 5.2
% Improving the Gradient Flow Step Scaling

The gradient flow is a continuous invertible smearing transformation that systematically removes short-distance lattice cut-off effects~\cite{Luscher:2009eq, Luscher:2010iy}.
At flow time $t = a^2 t_{\rm{lat}}$ it can be used to define a renormalized coupling at scale $\mu = 1 / \sqrt{8t}$
\begin{equation}
  \label{eq:def_g2}
  \gGF(\mu = 1 / \sqrt{8t}) = \frac{1}{\cN} \vev{t^2 E(t)},
\end{equation}
where ``$a$'' is the lattice spacing, $t_{\rm{lat}}$ is dimensionless, and the energy density $E(t) = -\frac{1}{2}\mbox{ReTr}\left[G_{\mu\nu}(t) G^{\mu\nu}(t)\right]$ is calculated at flow time $t$ with an appropriate lattice operator.
We evolve the gradient flow with the Wilson plaquette term and use the usual ``clover'' or ``symmetric'' definition of $G_{\mu\nu}(t)$.
The normalization \cN is set such that $\gGF(\mu)$ agrees with the continuum \MSbar coupling at tree level.

If the flow time is fixed relative to the lattice size, $\sqrt{8t} = cL$ with $c$ constant, the scale of the corresponding coupling $\gc(L)$ is set by the lattice size.
Like the well-known Schr\"odinger functional (SF) coupling, $\gc(L)$ can be used to compute a step scaling function~\cite{Fodor:2012td, Fodor:2012qh, Fritzsch:2013je}.
The greater flexibility of the gradient flow running coupling is a significant advantage over the more traditional SF coupling.
A single measurement of the gradient flow will provide \gc for a range of $c$.
In our study we obtain \gc for all $0 \leq c \leq 0.5$ separated by $\de t_{\rm{lat}} = 0.01$.
Each choice of $c$ corresponds to a different renormalization scheme, which can be explored simultaneously on the same set of configurations~\cite{Fritzsch:2013je}.

The normalization factor \cN in finite volume has been calculated for anti-periodic boundary conditions (BCs) in refs.~\cite{Fodor:2012td, Fodor:2012qh}, and for SF BCs in \refcite{Fritzsch:2013je}.
In this work we use anti-periodic BCs, for which
\begin{align}
  \frac{1}{\cN} & = \frac{128\pi^2}{3(N^2 - 1)(1 + \de(c))} &
  \de(c) & = \vartheta^4\left(e^{-1 / c^2}\right) - 1 - \frac{c^4 \pi^2}{3},
\end{align}
where $\vartheta(x) = \sum_{n = -\infty}^{\infty} x^{n^2}$ is the Jacobi elliptic function.
For $0 \leq c \leq 0.3$ the finite-volume correction $\de(c)$ computed in \refcite{Fodor:2012td} is small, $|\de(c)| \leq 0.03$.
As explained in refs.~\cite{Fodor:2012td, Fodor:2012qh}, the RG \be function of \gGF is two-loop universal with SF BCs, but only one-loop universal with anti-periodic BCs.

At non-zero lattice spacing \gGF has cut-off corrections.
These corrections could be $\cO(a)$ for unimproved actions, and even $\cO(a)$-improved actions could have large $\cO(a^2 [\log a]^n)$-type corrections~\cite{Balog:2009yj, Balog:2009np}.
In existing numerical studies of staggered or $\cO(a)$-improved Wilson fermions the leading lattice corrections appear to be $\cO(a^2)$~\cite{Fritzsch:2013je, Sommer:2014mea},
\begin{equation}
  \label{eq:lat_g2}
  \gGF(\mu; a) = \gGF(\mu; a = 0) + a^2 \cC + \cO(a^4 [\log a]^n, a^4).
\end{equation}
It is possible to remove, or at least greatly reduce, the $\cO(a^2)$ corrections in \eq{eq:lat_g2} by defining
\begin{equation}
  \label{eq:t-shift}
  \gtGF(\mu; a) = \frac{1}{\cN} \vev{t^2 E(t + \tau_0 a^2)}\, ,
\end{equation}
where $\tau_0 \ll t / a^2$ is a small shift in the flow time.
In the continuum limit $\tau_0 a^2 \to 0$ and $\gtGF(\mu) = \gGF(\mu)$.

There are several possible interpretations of the $t$-shift in \eq{eq:t-shift}.
The gradient flow is an invertible smearing transformation, so one can consider $\tau_0$ as an initial flow that does not change the IR properties of the system but leads to a new action.
The gradient flow coupling \gtGF in \eq{eq:t-shift} is calculated for this new action.
Alternatively one can consider the replacement of $\vev{t^2 E(t)}$ with $\vev{t^2 E(t + \tau_0 a^2)}$ as an improved operator for the energy density.
In either case the $t$-shift changes the $\cO(a^2)$ term of $\gGF(\mu; a)$.
If we expand $\gtGF(\mu)$ in $\tau_0 a^2$,
\begin{equation}
  \label{eq:expand}
  \gtGF(\mu; a) = \frac{1}{\cN} \vev{t^2 E(t)} + \frac{a^2 \tau_0}{\cN} \vev{t^2 \frac{\partial E(t)}{\partial t}},
\end{equation}
and choose $\tau_0$ such that the second term in \eq{eq:expand} cancels the $a^2 \cC$ term in \eq{eq:lat_g2}, we remove the leading lattice artifacts
\begin{equation}
  \gtopt(\mu; a) = \gGF(\mu; a = 0) + \cO(a^4 [\log a]^n, a^4).
\end{equation}
Full $\cO(a^2)$ improvement through a systematic improvement program would require adding terms to improve the flow equation, the action, the boundary conditions, and the energy density operator $\vev{t^2 E(t)}$~\cite{Sommer:2014mea}.
Since our proposed improvement involves only a single parameter $\tau_0$, this $\tau_0$ itself must depend on other parameters, most importantly on $\gtGF(\mu)$ and on the bare coupling through the lattice spacing dependence of the term $\vev{t^2 \frac{\partial E(t)}{\partial t}}$ in \eq{eq:expand}.
Optimizing $\tau_0$ both in the renormalized and bare couplings could remove the predictive power of the method.
Fortunately, as we will see in the next section, our numerical tests indicate that it is sufficient to choose $\tau_0$ to be a constant or only weakly $\gtGF(\mu)$ dependent to remove most $\cO(a^2)$ lattice artifacts.

Since the gradient flow is evaluated through numerical integration, the replacement $\gGF \to \gtGF$ can be done by a simple shift of $t$ without incurring any additional computational cost.
The optimal $t$-shift \topt can be identified by a simple procedure when the gradient flow is used for scale setting, which we will consider in a future publication.
In this chapter we concentrate on the step scaling function and find the \topt that removes the $\cO(a^2)$ terms of the discrete \be function corresponding to scale change $s$,
\begin{equation}
  \label{eq:beta_lat}
  \be_{\rm lat}(\gc; s; a) = \frac{\gtc(L; a) - \gtc(sL; a)}{\log(s^2)}.
\end{equation}

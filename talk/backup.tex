%********************%                    %********************%
                      %******************%
%********************%                    %********************%

\begin{frame}
  \addtocounter{framenumber}{-1}
  \frametitle{Auxiliary Slides}
\end{frame}

\begin{frame}
  \addtocounter{framenumber}{-1}
  \frametitle{Blocking}
  \begin{columns}[T]
    \begin{column}{.5\textwidth}
      \begin{block}{}
        \includegraphics[width=\textwidth]{fig/block2.png}
      \end{block}
    \end{column}
    \begin{column}{.5\textwidth}
      \begin{block}{}
        \begin{itemize}
          \item NHYP smear the lattice
          \item we use a scale factor $s=2$
          \item average links $U_{n,\mu}$, $U_{n+1,\mu}$
          \item perform all possible blocking
        \end{itemize}
      \end{block}
    \end{column}
  \end{columns}
\end{frame}

\begin{frame}
  \addtocounter{framenumber}{-1}
  \frametitle{Finite Volume Correction}
  \centering
  \includegraphics[width=\textwidth]{fig/3lm.png}
  \vspace{12pt}
  Can apply two separate two lattice matchings such the final blocked volumes for $n_b$, $n_{b-1}$, and $n_{b-2}$ are the same.
\end{frame}

\begin{frame}
  \frametitle{Gradient Flow - Origins}
  Use trivializing maps to improve the hybrid Monte Carlo (HMC) algorithm used in configuration generation\newline
  \vspace{12pt}
  Generate trivializing map by integrating a flow.
  \newline
  $U \xrightarrow{F^{-1}} V \xrightarrow{\text{HMC}} V' \xrightarrow{F} U'$
  \newline
  \begin{enumerate}
    \item rough links
    \item update links
    \item smooth links back
  \end{enumerate}
\end{frame}

\begin{frame}
  \frametitle{Wilson Flow - Definition}
    Wilson Action
    \begin{align*}
      S_w &= \frac{1}{g^2} \sum\limits_{p}\text{Re }\text{Tr}[\mathbb{I}-U(p)]
    \end{align*}
    Wilson Flow
    \begin{align*}
      V_t(x,\mu)|_{t=0} &= U(x,\mu) \\
      \dot{V}_t(x,\mu) &= -g_0^2[\nabla_{x,\mu}S_w(V_t)]V_t(x,\mu)
    \end{align*}

    The derivative is the natural $SU(3)$ valued differential operator with respect to $V_t(x,\mu)$.
\end{frame}

\begin{frame}
  \frametitle{Connection to Stout Smearing}
  \begin{center}
    \includegraphics[width=0.3\textwidth]{fig/stout.png}
  \end{center}
  \begin{columns}[T]
    \begin{column}{.5\textwidth}
      \begin{block}{}
        \begin{align*}
          Q = & \frac{i}{2}(\Sigma U^{\dagger}-U\Sigma^{\dagger}) \\
              & - \frac{i}{2N}\text{Tr}(\Sigma U^{\dagger}-U\Sigma^{\dagger})
        \end{align*}
        \begin{align*}
          W=e^{ipQ}U_{n,\mu}=(\mathbb{I}+iQ+...)U_{n,\mu}
        \end{align*}
      \end{block}
    \end{column}
    \begin{column}{.5\textwidth}
      \begin{block}{}
        \begin{itemize}
          \item $U$ is the link, $\Sigma$ is the staple sum
          \item Q is Hermitian
          \item Generates smeared links W that are in the group
        \end{itemize}
      \end{block}
    \end{column}
  \end{columns}
\end{frame}

\begin{frame}
  \frametitle{Stout Smearing as Wilson Flow}
  Integrating infinitesimal stout smearing steps is equivalent to finding the solution to the flow equations
  \begin{align*}
    U(x,\mu)\rightarrow U'(x,\mu)&=e^{\epsilon [Z(U)](x,\mu)}U(x,\mu) \\
    Z(U) &=-g^2_0\nabla_{x,\mu}S_w
  \end{align*}
\end{frame}


\begin{frame}
  \addtocounter{framenumber}{-1}
  \frametitle{Comparison}
  \begin{columns}[T]
    \begin{column}{0.5\textwidth}
      \begin{block}{\centering MCRG}
        \begin{itemize}
          \item does not depend on perturbation theory
          \item does not require special b.c.
          \item requires a scale factor of 2
        \end{itemize}
      \end{block}
    \end{column}
    \begin{column}{0.5\textwidth}
      \begin{block}{\centering Wilson Flow}
        \begin{itemize}
          \item unclear what happens on strong coupling side of IRFP
          \item does not require special b.c.
          \item does not require a scale factor of 2
        \end{itemize}
      \end{block}
    \end{column}
  \end{columns}
  \ \newline
  \begin{flushright}
    Fodor, Holland, Kuti, Nogradi, Wong: arXiv:1208.1051\\
    Fritzsch, Ramos:  arXiv: 1301.4388
  \end{flushright}
\end{frame}

\begin{frame}
  \addtocounter{framenumber}{-1}
  \frametitle{Wilson Flow - Preliminary} 
  \begin{columns}
    \begin{column}{0.7\linewidth}
      \includegraphics[width=1.2\textwidth]{fig/wflow.png}
    \end{column}
    \begin{column}{0.3\textwidth}
      $c=\frac{\sqrt{8t}}{L}=0.3$\vspace{24pt}
      $g_c^2=\frac{128\pi^2\left<t^2E(t)\right>}{3(N^2-1)(1+\delta_c)}$\vspace{24pt}
      $1+\delta_c\approx0.97$
      \begin{column}{0.3\textwidth}
      \end{column}
    \end{column}
  \end{columns}
\end{frame}

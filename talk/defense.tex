\documentclass{beamer}
%\usepackage{xcolor}
\usepackage{amsmath}
\usepackage{array}

\setbeamertemplate{navigation symbols}{}

\definecolor{renorm_traj}{HTML}{FF6A00}
\definecolor{irfp}{HTML}{FF1900}
\definecolor{uvfp}{HTML}{4950bd}
\definecolor{flow}{HTML}{63b092}


\definecolor{backgroundblue}{HTML}{E6F9FF}
\definecolor{backgroundorange}{HTML}{FFF3E6}
\definecolor{primary}{HTML}{00BFFF}
\definecolor{secondary}{HTML}{0056FF}
\definecolor{tertiary}{HTML}{00FF98}
\definecolor{complementary}{HTML}{FF8300}

\setbeamercolor{frametitle}{fg=complementary}
\setbeamercolor{title}{fg=complementary}
\setbeamercolor{author}{fg=secondary}
\setbeamercolor{background canvas}{fg=white, bg=white}

\setbeamercolor{palette primary}{fg=complementary, bg=complementary}
\setbeamercolor{palette secondary}{fg=secondary, bg=secondary}
\setbeamercolor{palette tertiary}{fg=tirtiary, bg=tertiary}

\setbeamertemplate{footline}[frame number]

\title{\huge{Improved Lattice Renormalization Group Techniques}}
\author{\textbf{Gregory Petropoulos}\\ Anqi Cheng, Anna Hasenfratz, David Schaich, Yuzhi Liu}
\institute{University of Colorado Boulder}
\date{December 17, 2014}

\AtBeginSection{\frame{\sectionpage}}
\AtBeginSubsection{\frame{\subsectionpage}}

\begin{document}
  {
    \setbeamertemplate{footline}{} 
    \begin{frame}
      \titlepage
    \end{frame}
  }
  \addtocounter{framenumber}{-1}

  \begin{frame}{Table of Contents}
    \tableofcontents
  \end{frame}
  \addtocounter{framenumber}{-1}

                      %******************%
%********************% SM and Beyond      %********************%
                      %******************%

  \section{Standard Model and Beyond}
  \addtocounter{framenumber}{-1}

  \begin{frame}
    \frametitle{The Standard Model}
    \centering
    \includegraphics[width=0.7\textwidth]{fig/higgs_standard_mod.png}
  \end{frame}

  \begin{frame}
    \frametitle{Hierarchy Problem}
    \centering
    $m^2_H=m^2_0+\frac{3}{4\pi}\lambda\Lambda_{cut}^2$
  \end{frame}

  \begin{frame}
    \frametitle{New Strong Dynamics?}
    \begin{itemize}
      \item Asymptotically free, broken chiral symmetry, gauge-fermion quantum field theory
      \item Goldstone bosons from the broken chiral symmetry
      \item Higgs:  $0^{++}$
      \item Dilaton?
      \item Higgs must be light (126 GeV)
      \item other bound states must be heavy
    \end{itemize}
  \end{frame}

  \begin{frame}
    \frametitle{Walking}
    \centering
  \end{frame}

  \begin{frame}
    \frametitle{The $\beta$ Function}
    \centering
    \begin{align*}
      \beta(g^2)&=\frac{d g^2}{d\text{log}(\mu^2)}=-\frac{b_1}{16\pi^2}g^4-\frac{b_2}{(16\pi^2)^2}g^6
    \end{align*}
    \vspace{10pt}
    \tiny{$b_1$ and $b_2$ depend on the gauge group, $N_f$ and fermionic representation}
    \vspace{10pt}
    \centering
    \includegraphics[width=0.6\textwidth]{fig/betafn.png}
  \end{frame}

                      %******************%
%********************% Lattice Studies    %********************%
                      %******************%

  \section{Lattice Studies}
  \addtocounter{framenumber}{-1}

  \begin{frame}
    \frametitle{Lattice Gauge Theory}
    \centering
  \end{frame}

  \begin{frame}
    \frametitle{Lattice BSM Studies}
    \centering
    \includegraphics[width=0.9\textwidth]{fig/lattice_studies.pdf}
  \end{frame}

  \begin{frame}
    \frametitle{Studying the $\beta$ Function on the Lattice}
    \begin{center}
      There are several techniques:
    \end{center}
    \begin{itemize}
      \item Schrodinger Functional
      \item Twisted Polyakov Loop scheme
      \item MCRG
      \item Wilson Flow Step Scaling
      \item Wilson Flow Step Scaling + Schrodinger
      \item Wilson Flow MCRG
    \end{itemize}
  \end{frame}

                      %******************%
%********************% Wilson RG          %********************%
                      %******************%

  \section{Wilson Renormalization Group}
  \addtocounter{framenumber}{-1}

  \begin{frame}
    \frametitle{Wilson Renormalization Group}
    RG transformation integrates out high momentum modes
    \begin{columns}[T]
      \begin{column}{.5\textwidth}
        \begin{block}{}
          \includegraphics[width=\textwidth]{fig/renorm_uvfp.png}
        \end{block}
      \end{column}
      \begin{column}{.5\textwidth}
        \begin{block}{}
          \includegraphics[width=\textwidth]{fig/renorm_irfp.png}
        \end{block}
      \end{column}
    \end{columns}
    \begin{itemize}
      \item \emph{\textcolor{flow}{flow}} to \textcolor{renorm_traj}{renormalized trajectory} in irrelevant directions
      \item \emph{\textcolor{flow}{flow}} along \textcolor{renorm_traj}{renormalized trajectory} in relevant directions
      \item \emph{\textcolor{flow}{flow}} away from \textcolor{uvfp}{ultraviolet fixed points}
      \item \emph{\textcolor{flow}{flow}} to \textcolor{irfp}{infrared fixed points}
    \end{itemize}
  \end{frame}

  \begin{frame}
    \frametitle{Monte Carlo Renormalization Group}
    \begin{columns}[T]
      \begin{column}{.5\textwidth}
        \begin{block}{}
          \includegraphics[width=\textwidth]{fig/mcrg_uvfp.png}
        \end{block}
      \end{column}
      \begin{column}{.5\textwidth}
        \begin{block}{}
          \begin{itemize}
            \item block
            \item match $(\beta_1,n_b)\equiv(\beta_2,n_{b-1})$
            \item optimize
          \end{itemize}
        \end{block}
      \end{column}
    \end{columns}
  \end{frame}

  \begin{frame}
    \frametitle{Blocking}
    \begin{columns}[T]
      \begin{column}{.5\textwidth}
        \begin{block}{}
          \includegraphics[width=\textwidth]{fig/block.png}
        \end{block}
      \end{column}
      \begin{column}{.5\textwidth}
        \begin{block}{}
          \begin{itemize}
            \item integrate out UV
            \item reduce degrees of freedom by a factor of $s^D$
            \item $s$ - scale factor
            \item $D$ - dimension
            \item $a \rightarrow 2a$
            \item $\Lambda_{\text{cut}} \rightarrow \tfrac{\Lambda_{\text{cut}}}{2}$
            \item flow in parameter space towards and along RT
          \end{itemize}
        \end{block}
      \end{column}
    \end{columns}
  \end{frame}

  \begin{frame}
    \frametitle{Two Lattice Matching}
    \centering
    \includegraphics[width=0.5\textwidth]{fig/2lm.png}
  \end{frame}

  \begin{frame}
    \frametitle{Two Lattice Matching}
    \centering
    \includegraphics[width=0.5\textwidth]{fig/mcrg_match.png}
  \end{frame}

  \begin{frame}
    \frametitle{Two Lattice Matching}
    \begin{columns}[T]
      \begin{column}{.5\textwidth}
        \begin{block}{}
          \includegraphics[width=\textwidth]{fig/observables.png}
        \end{block}
      \end{column}
      \begin{column}{.5\textwidth}
        \begin{block}{}
          \begin{itemize}
            \item The actions will be the same when all the observables are the same
            \item measuring small observables allows us to block down to small lattices
            \item measure plaquette, 6 link loops, 8 link loop
          \end{itemize}
        \end{block}
      \end{column}
    \end{columns}
  \end{frame}

  \begin{frame}
    \frametitle{Need for Optimization}
    \begin{columns}[T]
      \begin{column}{.5\textwidth}
        \begin{block}{}
          \includegraphics[width=\textwidth]{fig/need_to_opt.png}
        \end{block}
      \end{column}
      \begin{column}{.5\textwidth}
        \begin{block}{}
          \begin{itemize}
            \item you are not guaranteed to reach the renormalized trajectory
            \item each time you block $L\rightarrow\frac{L}{s}$
            \item to reach RT with $V=L^4$ in $n$ steps you need $V=s^nL^{4}$
            \item to reach RT (left) with $V=2^4$ need $V=512^4$!
          \end{itemize}
        \end{block}
      \end{column}
    \end{columns}
  \end{frame}

  \begin{frame}
    \frametitle{Traditional Optimization}
    \begin{columns}[T]
      \begin{column}{.5\textwidth}
        \begin{block}{}
          \includegraphics[width=\textwidth]{fig/trad_opt.png}
        \end{block}
      \end{column}
      \begin{column}{.5\textwidth}
        \begin{block}{}
          \begin{itemize}
            \item change the block transformation at each large volume bare coupling
            \item move the RT in coupling space
            \item NHYP parameters ($\alpha$, 0.2, 0.2)
          \end{itemize}
        \end{block}
      \end{column}
    \end{columns}
  \end{frame}

  \begin{frame}
    \frametitle{Results}
    Summary of calculation and lattices used.
  \end{frame}

  \begin{frame}
    \frametitle{Results - 8 flavors}
    \centering
    \includegraphics[width=0.7\textwidth]{fig/8flavor.pdf}
  \end{frame}

  \begin{frame}
    \frametitle{Results - 12 flavors}
    \centering
    \includegraphics[width=0.7\textwidth]{fig/12flavor.pdf}
  \end{frame}

  \begin{frame}
    \frametitle{Stitching Together different $s_b$}
    \begin{columns}[T]
      \begin{column}{.5\textwidth}
        \begin{block}{}
          \includegraphics[width=.8\textwidth]{fig/opt_prob.png}
        \end{block}
      \end{column}
      \begin{column}{.5\textwidth}
        \begin{block}{}
          \begin{itemize}
            \item change the block transformation at each large volume bare coupling
            \item move the RT in coupling space
            \item location of IRFP is scheme dependant
            \item `stitching together' different step scaling functions
          \end{itemize}
        \end{block}
      \end{column}
    \end{columns}
  \end{frame}

  \begin{frame}
    \frametitle{Alternative Optimizations}
    \begin{columns}[T]
      \begin{column}{.4\textwidth}
        \begin{block}{}
          \includegraphics[width=\textwidth]{fig/wilson_flow_opt.png}
        \end{block}
      \end{column}
      \begin{column}{.6\textwidth}
        \begin{block}{}
          Simulate close to RT (hard)\newline
          Find optimization that:
          \begin{itemize}
            \item does not change lattice spacing
            \item moves the system toward the RT
          \end{itemize}
          Once on RT proceed with blocking
        \end{block}
      \end{column}
    \end{columns}
  \end{frame}

                      %******************%
%********************% Gradient Flow      %********************%
                      %******************%

  \section{Gradient (Wilson) Flow}
  \addtocounter{framenumber}{-1}

  \begin{frame}
    \frametitle{Gradient Flow - Origins}
    Use trivializing maps to improve the HMC algorithm used in configuration generation\newline
    Generate trivializing map by integrating a flow.
    \newline
    $U \xrightarrow{F^{-1}} V \xrightarrow{\text{HMC}} V' \xrightarrow{F} U'$
    \newline
    \begin{enumerate}
      \item rough links
      \item update links
      \item smooth links back
    \end{enumerate}
  \end{frame}

  \begin{frame}
    \frametitle{Wilson Flow - Definition}
      Wilson Action
      \begin{align*}
        S_w &= \frac{1}{g^2} \sum\limits_{p}\text{Re }\text{Tr}[\mathbb{I}-U(p)]
      \end{align*}
      Wilson Flow
      \begin{align*}
        V_t(x,\mu)|_{t=0} &= U(x,\mu) \\
        \dot{V}_t(x,\mu) &= -g_0^2[\nabla_{x,\mu}S_w(V_t)]V_t(x,\mu)
      \end{align*}

      The derivative is the natural $SU(3)$ valued differential operator with respect to $V_t(x,\mu)$.
  \end{frame}

  \begin{frame}
    \frametitle{Connection to Stout Smearing}
    \begin{center}
      \includegraphics[width=0.3\textwidth]{fig/stout.png}
    \end{center}
    \begin{columns}[T]
      \begin{column}{.5\textwidth}
        \begin{block}{}
          \begin{align*}
            Q = & \frac{i}{2}(\Sigma U^{\dagger}-U\Sigma^{\dagger}) \\
                & - \frac{i}{2N}\text{Tr}(\Sigma U^{\dagger}-U\Sigma^{\dagger})
          \end{align*}
          \begin{align*}
            W=e^{ipQ}U_{n,\mu}=(\mathbb{I}+iQ+...)U_{n,\mu}
          \end{align*}
        \end{block}
      \end{column}
      \begin{column}{.5\textwidth}
        \begin{block}{}
          \begin{itemize}
            \item $U$ is the link, $\Sigma$ is the staple sum
            \item Q is Hermitian
            \item Generates smeared links W that are in the group
          \end{itemize}
        \end{block}
      \end{column}
    \end{columns}
  \end{frame}

  \begin{frame}
    \frametitle{Stout Smearing as Wilson Flow}
    Integrating infinitesimal stout smearing steps is equivalent to finding the solution to the flow equations
    \begin{align*}
      U(x,\mu)\rightarrow U'(x,\mu)&=e^{\epsilon [Z(U)](x,\mu)}U(x,\mu) \\
      Z(U) &=-g^2_0\nabla_{x,\mu}S_w
    \end{align*}
  \end{frame}

  \begin{frame}
    \frametitle{Properties of the Gradient Flow}
    \begin{itemize}
      \item Forward flow smooths out fluctuations
      \item The flow does not change IR physics
      \item The flow is generated by infinitesimal stout smearing steps
    \end{itemize}
  \end{frame}

  \subsection{Application to MCRG}
  \addtocounter{framenumber}{-1}

  \begin{frame}
    \frametitle{Optimization with Gradient Flow}
    Wilson flow integrates infinitesimal smearing steps.
    \begin{columns}[T]
      \begin{column}{.4\textwidth}
        \begin{block}{}
          \includegraphics[width=\textwidth]{fig/wilson_flow_opt.png}
        \end{block}
      \end{column}
      \begin{column}{.6\textwidth}
        \begin{block}{}
          \begin{itemize}
            \item Wilson flow removes UV fluctuation
            \item Wilson flow does not change lattice spacing
            \item Moves system toward the RT
            \item Proceed with MCRG
          \end{itemize}
        \end{block}
      \end{column}
    \end{columns}
  \end{frame}

  \begin{frame}
    \frametitle{SU(3) $\text{N}_{\text{f}}=$ 12}
    \begin{columns}[c]
      \begin{column}{0.5\textwidth}
        \begin{itemize}
          \item fermion mass exactly zero
          \item APBC lattices
          \item fundamental adjoint gauge action
          \item nHYP smeared staggered fermions
        \end{itemize}
      \end{column}
      \begin{column}{0.5\textwidth}
        \begin{tabular}{| r | l |}\hline
          Volume & Bare Coupling ($\beta_F$) \\\hline
          $6^4$  & 3.4, 3.6, ... 7.8, 8.0 \\
          $8^4$  & 3.0, 3.2, ... 7.8, 8.0 \\
          $12^4$ & 3.4, 3.6, ... 7.8, 8.0 \\
          $16^4$ & 3.0, 3.2, ... 7.8, 8.0 \\
          $24^4$ & 4.0, 4.5, ... 7.5, 8.0 \\
          $32^4$ & 4.0, 5.0, ... 5.5, 6.0 \\
          $48^4$ & 6.0 \\ \hline
        \end{tabular}
      \end{column}
    \end{columns}
  \end{frame}

  \begin{frame}
    \frametitle{$6^4$ - $12^4$ - $24^4$ Matching}
    \begin{center}
      Errors indicate the spread in $s_b$ predicted by each observable.
    \end{center}
    \begin{columns}[T]
      \begin{column}{.3\textwidth}
        \begin{block}{}
          \begin{itemize}
            \item nHYP block transformation
            \item $n_b=3$
            \item $V_f=3^4$
          \end{itemize}
        \end{block}
      \end{column}
      \begin{column}{.7\textwidth}
        \begin{block}{}
          \includegraphics[width=\textwidth]{fig/12flav_6-12-24_3lm.pdf}
        \end{block}
      \end{column}
    \end{columns}
    \begin{center}
      \only<1>{
        Scheme 1:  \textcolor{red}{0.6 0.2 0.2}\quad\quad Scheme 2:  0.6 0.3 0.2 \\
        Scheme 3:  \textcolor{blue}{0.65 0.3 0.2}
      }
    \end{center}
  \end{frame}

  \begin{frame}
    \frametitle{$8^4$ - $16^4$ - $32^4$ Matching}
    \begin{center}
      Errors indicate the spread in $s_b$ predicted by each observable.
    \end{center}
    \begin{columns}[T]
      \begin{column}{.3\textwidth}
        \begin{block}{}
          \begin{itemize}
            \item nHYP block transformation
            \item $n_b=3$
            \item $V_f=4^4$
          \end{itemize}
        \end{block}
      \end{column}
      \begin{column}{.7\textwidth}
        \begin{block}{}
          \includegraphics[width=\textwidth]{fig/12flav_8-16-32_44_as_3lm.pdf}
        \end{block}
      \end{column}
    \end{columns}
    \begin{center}
      \only<1>{
        Scheme 1:  \textcolor{red}{0.6 0.2 0.2}\quad\quad Scheme 2:  0.6 0.3 0.2 \\
        Scheme 3:  \textcolor{blue}{0.65 0.3 0.2}
      }
      \only<2>{
        \begin{exampleblock}{}
          Now let us look at Scheme 1 and change $n_b$, $V_f$.
        \end{exampleblock}
      }
    \end{center}
  \end{frame}

  \begin{frame}
    \frametitle{$8^4$ - $16^4$ - $32^4$ Matching}
    \begin{columns}[T]
      \begin{column}{.3\textwidth}
        \begin{block}{}
          \begin{itemize}
            \item Scheme 1
            \item $n_b=3$ $V_f=3^4$
            \item \textcolor{blue}{$n_b=3$ $V_f=4^4$}
          \end{itemize}
        \end{block}
      \end{column}
      \begin{column}{.7\textwidth}
        \begin{block}{}
          \includegraphics[width=\textwidth]{fig/12flav_8-16-32_3lm_44.pdf}
        \end{block}
      \end{column}
    \end{columns}
  \end{frame}

  \begin{frame}
    \frametitle{$8^4$ - $16^4$ - $32^4$ Matching}
    \begin{columns}[T]
      \begin{column}{.3\textwidth}
        \begin{block}{}
          \begin{itemize}
            \item Scheme 1
            \item $n_b=3$ $V_f=3^4$
            \item \textcolor{blue}{$n_b=3$ $V_f=4^4$}
            \item \textcolor{red}{$n_b=4$ $V_f=2^4$}
          \end{itemize}
        \end{block}
      \end{column}
      \begin{column}{.7\textwidth}
        \begin{block}{}
          \includegraphics[width=\textwidth]{fig/12flav_8-16-32_3lm_22.pdf}
        \end{block}
      \end{column}
    \end{columns}
  \end{frame}

  \begin{frame}
    \frametitle{$12^4$ - $24^4$ - $48^4$ Matching}
    \begin{columns}[T]
      \begin{column}{.3\textwidth}
        \begin{block}{}
          \begin{itemize}
            \item Scheme 1
            \item \textcolor{red}{$n_b=3$ $V_f=3^4$}
            \item $n_b=3$ $V_f=4^4$
            \item \textcolor{blue}{$n_b=4$ $V_f=2^4$}
            \item \textcolor{magenta}{$n_b=3$ $V_f=6^4$}
            \item \textcolor{cyan}{$n_b=4$ $V_f=3^4$}
          \end{itemize}
        \end{block}
      \end{column}
      \begin{column}{.7\textwidth}
        \begin{block}{}
          \includegraphics[width=\textwidth]{fig/12flav_12-24-48_3lm.pdf}
        \end{block}
      \end{column}
    \end{columns}
  \end{frame}

  \begin{frame}
    \frametitle{Summary - WMCRG}
    \begin{itemize}
      \item New work on an established method
      \item With Wilson flow MCRG we can find a unique step scaling function
      \item Wilson flow MCRG predicts an IRFP for SU(3) $N_f=12$ in all analysis
      \item Wilson flow MCRG is computationally inexpensive; does not rely on perturbation theory
    \end{itemize}
  \end{frame}

  \subsection{Wilson Flow Step Scaling Function}
  \addtocounter{framenumber}{-1}

  \begin{frame}
    \frametitle{Another Approach}
    New RG approaches have been proposed that use Wilson Flow to find the renormalized step scaling function.
    \begin{itemize}
      \item $g_c^2(\mu)$
      \item $\mu = \frac{1}{\sqrt{8t}}$
      \item $t = \frac{(cL)^2}{8} \rightarrow \mu = \frac{1}{cL}$ scale is set by the volume
      \item From gradient flow define renormalized coupling $g_c^2=\frac{128\pi^2\left<t^2E(t)\right>}{3(N^2-1)(1+\delta_c)}$
      \item similar to Schrodinger functional
    \end{itemize}
  \end{frame}

  \begin{frame}
    \frametitle{First Look}
    Very different from asymptotic freedom.
    \begin{columns}
      \begin{column}{0.7\textwidth}
        \includegraphics[width=1.2\textwidth]{fig/wflow.png}
      \end{column}
      \begin{column}{0.3\textwidth}
        \begin{centering}
          \vspace{12pt}
          fixed $\beta$ in strong coupling:\\
          $g^2(L)$ decreases as\\
          $L$ increases\\
          crossigns=IRFP\\
        \end{centering}
      \end{column}
    \end{columns}
  \end{frame}

  \begin{frame}
    \frametitle{Continuum Extrapolation}
    This is not $\beta = \infty$ continuum limit.
    \begin{columns}
      \begin{column}{0.7\textwidth}
        \includegraphics[width=1.2\textwidth]{fig/wflow_continuum.png}
      \end{column}
      \begin{column}{0.3\textwidth}
        \begin{center}
          fix $g^2(L)=g_0^2$\\\vspace{24pt}
          compare\\$g^2(2L)-g^2(L)$\\\vspace{24pt}
          $L\rightarrow \infty$\\requires\\decreasing $\beta$
        \end{center}
      \end{column}
    \end{columns}
  \end{frame}

  \begin{frame}
    \frametitle{Continuum Extrapolation}
    \begin{columns}
      \begin{column}{0.7\textwidth}
        \includegraphics[width=1.2\textwidth]{fig/cont_extrap_no_improv.png}
      \end{column}
      \begin{column}{0.3\textwidth}
        \begin{center}
        \end{center}
      \end{column}
    \end{columns}
  \end{frame}

  \begin{frame}
    \frametitle{Wilson Flow Optimization}
  \end{frame}

  \begin{frame}
    \frametitle{Combined Fit - Vary $\tau_0$}
      \includegraphics[width=\textwidth]{fig/fit_s2_c02.pdf}
  \end{frame}

  \begin{frame}
    \frametitle{Combined Fit - $\tau_{opt}$}
    \includegraphics[width=\textwidth]{fig/fit_c02_tau004.pdf}
  \end{frame}

  \begin{frame}
    \frametitle{Summary - Optimized Gradient Flow Step Scaling}
    \begin{itemize}
      \item New work on an established method
      \item With Wilson flow MCRG we can find a unique step scaling function
      \item Wilson flow MCRG predicts an IRFP for SU(3) $N_f=12$ in all analysis
      \item Wilson flow MCRG is computationally inexpensive; does not rely on perturbation theory
    \end{itemize}
  \end{frame}

  \begin{frame}
    \frametitle{Conclusions}
    \begin{itemize}
      \item Wilson flow step scaling is a complementary technique to MCRG
      \item Our preliminary results are consistent with what we see in MCRG
      \item Find IRFP (crossings) and backwards flow on strong coupling side of IRFP
    \end{itemize}
  \end{frame}

  \begin{frame}
    \frametitle{Acknowledgments}
    \begin{itemize}
      \item{Anna, David, Anqi, Yuzhi}
      \item{DOE Office of Science Graduate Student Fellowship}
      \item{USQCD}
      \item{University of Colorado Research Computing (Janus)}
      \item{NSF XSEDE}
      \item{Thank you for watching}
    \end{itemize}
  \end{frame}

%********************%                    %********************%
                      %******************%
%********************%                    %********************%

  \begin{frame}
    \addtocounter{framenumber}{-1}
    \frametitle{Auxiliary Slides}
  \end{frame}

  \begin{frame}
    \addtocounter{framenumber}{-1}
    \frametitle{Blocking}
    \begin{columns}[T]
      \begin{column}{.5\textwidth}
        \begin{block}{}
          \includegraphics[width=\textwidth]{fig/block2.png}
        \end{block}
      \end{column}
      \begin{column}{.5\textwidth}
        \begin{block}{}
          \begin{itemize}
            \item NHYP smear the lattice
            \item we use a scale factor $s=2$
            \item average links $U_{n,\mu}$, $U_{n+1,\mu}$
            \item perform all possible blocking
          \end{itemize}
        \end{block}
      \end{column}
    \end{columns}
  \end{frame}

  \begin{frame}
    \addtocounter{framenumber}{-1}
    \frametitle{Finite Volume Correction}
    \centering
    \includegraphics[width=\textwidth]{fig/3lm.png}
    \vspace{12pt}
    Can apply two separate two lattice matchings such the final blocked volumes for $n_b$, $n_{b-1}$, and $n_{b-2}$ are the same.
  \end{frame}

  \begin{frame}
    \addtocounter{framenumber}{-1}
    \frametitle{Comparison}
    \begin{columns}[T]
      \begin{column}{0.5\textwidth}
        \begin{block}{\centering MCRG}
          \begin{itemize}
            \item does not depend on perturbation theory
            \item does not require special b.c.
            \item requires a scale factor of 2
          \end{itemize}
        \end{block}
      \end{column}
      \begin{column}{0.5\textwidth}
        \begin{block}{\centering Wilson Flow}
          \begin{itemize}
            \item unclear what happens on strong coupling side of IRFP
            \item does not require special b.c.
            \item does not require a scale factor of 2
          \end{itemize}
        \end{block}
      \end{column}
    \end{columns}
    \ \newline
    \begin{flushright}
      Fodor, Holland, Kuti, Nogradi, Wong: arXiv:1208.1051\\
      Fritzsch, Ramos:  arXiv: 1301.4388
    \end{flushright}
  \end{frame}

  \begin{frame}
    \addtocounter{framenumber}{-1}
    \frametitle{Wilson Flow - Preliminary} 
    \begin{columns}
      \begin{column}{0.7\linewidth}
        \includegraphics[width=1.2\textwidth]{fig/wflow.png}
      \end{column}
      \begin{column}{0.3\textwidth}
        $c=\frac{\sqrt{8t}}{L}=0.3$\vspace{24pt}
        $g_c^2=\frac{128\pi^2\left<t^2E(t)\right>}{3(N^2-1)(1+\delta_c)}$\vspace{24pt}
        $1+\delta_c\approx0.97$
        \begin{column}{0.3\textwidth}
        \end{column}
      \end{column}
    \end{columns}
  \end{frame}

\end{document}

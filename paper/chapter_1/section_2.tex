% 1.2
% Organizaation

I grappeled with how to organize this thesis for some time.
I finally decided that it would be easiest for the majority of the readers as well as myself if follow the chronology of my work.

Before delving into my own work, I use chapter 2 to set the stage.
In this chapter I discuss the reasons that the system I primarily study are interesting and why it is important to develop new techniques for  lattice studies.
I also use for In Chapter 2 I discuss the motivations that lead both to this thesis and the broader effort by the Lattice community.

Chapter 3 provides the reader with a brief overview of what goes into a lattice field theory calculation.  
This topic is a subject of many books, my goal is to provide the reader with a flavor of how a lattice caluclation is performed and highlight areas that are useful in later sections.

%Chapter 4 is an overview of the theoretical underpinnings of the techniques I use to calculate the step scaling funciton.  
%I open with a discussion of the beta function and then proceed with an overview of the renormalization group.
%The first lattice technique I describe is the Monte Carlo Renormalization Group (MCRG).
%I then introduce the Wilson Flow and methods to obtain the beta function from the Wilson Flow.
%Finally I show how the Wilson Flow can be used to improve MCRG.

Chapter 5 puts chapter 4 into practice.
I provide the details of our calculations and present our results.
For consistency I discuss MCRG first followed by Wilson Flow and finally Wilson Flow MCRG.

Chapter 6

In chapter 7 I make some final remarks about interpreting our results.
I will also summarize other methods that our group used to study this system.
As expected all of our results show that SU(3) gauge theories with 12 chiral fermions are conformal.

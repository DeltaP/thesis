                      %******************%
%********************% Wilson RG          %********************%
                      %******************%

  \section{Wilson Renormalization Group}
  \addtocounter{framenumber}{-1}

  \begin{frame}
    \frametitle{Wilson Renormalization Group}
    RG transformation integrates out high momentum modes
    \begin{columns}[T]
      \begin{column}{.5\textwidth}
        \begin{block}{}
          \includegraphics[width=\textwidth]{fig/renorm_uvfp.png}
        \end{block}
      \end{column}
      \begin{column}{.5\textwidth}
        \begin{block}{}
          \includegraphics[width=\textwidth]{fig/renorm_irfp.png}
        \end{block}
      \end{column}
    \end{columns}
    \begin{itemize}
      \item \emph{\textcolor{flow}{flow}} to \textcolor{renorm_traj}{renormalized trajectory} in irrelevant directions
      \item \emph{\textcolor{flow}{flow}} along \textcolor{renorm_traj}{renormalized trajectory} in relevant directions
      \item \emph{\textcolor{flow}{flow}} away from \textcolor{uvfp}{ultraviolet fixed points}
      \item \emph{\textcolor{flow}{flow}} to \textcolor{irfp}{infrared fixed points}
    \end{itemize}
  \end{frame}

  \begin{frame}
    \frametitle{Monte Carlo Renormalization Group}
    Goal:  Calculate the bare step scaling function, $s_b$
    \vspace{24pt}
    \begin{columns}[T]
      \begin{column}{.5\textwidth}
        \begin{block}{}
          \includegraphics[width=\textwidth]{fig/mcrg_uvfp.png}
        \end{block}
      \end{column}
      \begin{column}{.5\textwidth}
        \begin{itemize}
          \item block
          \item match
          \begin{equation*}
            \begin{aligned}
              (\beta_1,n_b) &\equiv(\beta_2,n_{b-1})\\
              \Delta\beta &\equiv \beta_1-\beta_2\\
              \lim_{n_b\to\infty} \Delta\beta &= s_b
            \end{aligned}
          \end{equation*}
          \item optimize
        \end{itemize}
      \end{column}
    \end{columns}
  \end{frame}

  \begin{frame}
    \frametitle{Blocking Procedure}
    \begin{columns}[T]
      \begin{column}{.5\textwidth}
        \begin{block}{}
          \includegraphics[width=\textwidth]{fig/block.png}
        \end{block}
      \end{column}
      \begin{column}{.5\textwidth}
        \begin{block}{}
          \begin{itemize}
            \item integrate out UV
            \item flow in parameter space towards and along RT
            \item nHYP smearing in block transformation
            \vspace{12pt}
            \item reduce degrees of freedom by a factor of $s^D$
            \item $s$ - scale factor
            \item $D$ - dimension
            \item $a \rightarrow 2a$
            \item $\Lambda_{\text{cut}} \rightarrow \tfrac{\Lambda_{\text{cut}}}{2}$
          \end{itemize}
        \end{block}
      \end{column}
    \end{columns}
  \end{frame}

  \begin{frame}
    \frametitle{Two Lattice Matching in Action Space}
    \centering
    \includegraphics[width=0.5\textwidth]{fig/2lm.png}
    \newline
    \vspace{12pt}
    \emph{Hard to calculate and match blocked lattice actions!}
  \end{frame}

  \begin{frame}
    \frametitle{Two Lattice Matching on the Lattice}
    \begin{columns}[T]
      \begin{column}{.4\textwidth}
        \includegraphics[width=\textwidth]{fig/mcrg_match.png}
      \end{column}
      \begin{column}{.6\textwidth}
        \begin{itemize}
          \item If $S_1=S_2$ then $<O_1>=<O_2>$\\
          \item Blocking ensemble gives correct Boltzmann weights for blocked latices
        \end{itemize}
        \vspace{32pt}
      \begin{center}\emph{What observables to measure?}\end{center}
      \end{column}
    \end{columns}
  \end{frame}

  \begin{frame}
    \frametitle{Two Lattice Matching Observables}
    \begin{columns}[T]
      \begin{column}{.5\textwidth}
        \begin{figure}
          \includegraphics[width=\textwidth]{fig/observables.png}
          \caption{Observables}
        \end{figure}
      \end{column}
      \begin{column}{.5\textwidth}
        \begin{block}{}
          \begin{itemize}
            \item The actions will be the same when all the observables are the same
            \item Measuring small observables allows us to block down to small lattices
            \item Measure plaquette, 6 link loops, 8 link loop
            \item All observables should give the same $s_b$, differences reflect systematic error
          \end{itemize}
        \end{block}
      \end{column}
    \end{columns}
  \end{frame}

  \begin{frame}
    \frametitle{Need for Optimization to reach RT}
    \begin{columns}[T]
      \begin{column}{.5\textwidth}
        \begin{block}{}
          \includegraphics[width=\textwidth]{fig/need_to_opt.png}
        \end{block}
      \end{column}
      \begin{column}{.5\textwidth}
        \begin{block}{}
          \begin{itemize}
            \item you are not guaranteed to reach the renormalized trajectory
            \item each time you block $L\rightarrow\frac{L}{s}$
            \item to reach RT with $V=L^4$ in $n$ steps you need $V=s^nL^{4}$
            \item to reach RT (left) with $V=2^4$ need $V=512^4$!
          \end{itemize}
        \end{block}
      \end{column}
    \end{columns}
  \begin{center}\emph{How can we optimize our block transformation?}\end{center}
  \end{frame}

  \begin{frame}
    \frametitle{Optimizing the Block Transformation}
    \begin{columns}[T]
      \begin{column}{.5\textwidth}
        \begin{block}{}
          \includegraphics[width=\textwidth]{fig/trad_opt.png}
        \end{block}
      \end{column}
      \begin{column}{.5\textwidth}
        \begin{block}{}
          \begin{itemize}
            \item change the block transformation at each large volume bare coupling
            \item nHYP parameters ($\alpha$, 0.2, 0.2)
            \item move the RT in coupling space
          \end{itemize}
        \end{block}
      \end{column}
    \end{columns}

  \end{frame}

  \begin{frame}
    \frametitle{Results}
    Summary of calculation and lattices used.
  \end{frame}

  \begin{frame}
    \frametitle{Results - 8 flavors}
    \begin{columns}[T]
      \begin{column}{.6\textwidth}
        \includegraphics[width=\textwidth]{fig/8flavor.pdf}
      \end{column}
      \begin{column}{.4\textwidth}
        No IRFP\\
        Chirally Broken\\
        Confining\\
        \vspace{48pt}
        $\ensuremath{\cancel{S^4}}$ lattice artifact
      \end{column}
    \end{columns}
    \vspace{12pt}
    Error bars show spread in $s_b$ predicted by each observable.
  \end{frame}

  \begin{frame}
    \frametitle{Results - 12 flavors}
    \begin{columns}[T]
      \begin{column}{.6\textwidth}
        \includegraphics[width=\textwidth]{fig/12flavor.pdf}
      \end{column}
      \begin{column}{.4\textwidth}
        Conformal:\\
        \vspace{12pt}
        \begin{tabular}{l l}
          Weak Coupling: & $s_b>0$ \\
          Strong Coupling: & $s_b\lessapprox0$\\
        \end{tabular}
        \newline\\
        \vspace{24pt}
        $\ensuremath{\cancel{S^4}}$ lattice artifact
      \end{column}
    \end{columns}
    \vspace{12pt}
    Error bars show spread in $s_b$ predicted by each observable.\\
    \vspace{12pt}
    \emph{Why isn't the fixed point more clear?}
  \end{frame}

  \begin{frame}
    \frametitle{Stitching Together different $s_b$}
    \begin{columns}[T]
      \begin{column}{.5\textwidth}
        \begin{block}{}
          \includegraphics[width=.8\textwidth]{fig/opt_prob.png}
        \end{block}
      \end{column}
      \begin{column}{.5\textwidth}
        \begin{block}{}
          \begin{itemize}
            \item change the block transformation at each large volume bare coupling
            \item move the RT in coupling space
            \item location of IRFP is scheme dependant
            \item `stitching together' different step scaling functions
          \end{itemize}
        \end{block}
        \begin{center}\emph{Can we do better?}\end{center}
      \end{column}
    \end{columns}
  \end{frame}

  \begin{frame}
    \frametitle{Alternative Optimizations}
    \begin{columns}[T]
      \begin{column}{.4\textwidth}
        \begin{block}{}
          \includegraphics[width=\textwidth]{fig/wilson_flow_opt.png}
        \end{block}
      \end{column}
      \begin{column}{.6\textwidth}
        \begin{block}{}
          Simulate close to RT (hard)\newline
          Find optimization that:
          \begin{itemize}
            \item does not change lattice spacing
            \item moves the system toward the RT
          \end{itemize}
          Once on RT proceed with blocking
        \end{block}
      \end{column}
    \end{columns}
  \end{frame}

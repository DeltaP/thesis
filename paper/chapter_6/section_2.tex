% 6.2
% MCRG recap?

MCRG techniques probe lattice field theories by applying RG blocking transformations that integrate out high-momentum (short-distance) modes, moving the system in the infinite-dimensional space of lattice-action couplings.
In an IR-conformal system on the $m = 0$ critical surface, a renormalized trajectory runs from the perturbative gaussian FP (where the gauge coupling \be is a relevant operator) to the IRFP (where \be is irrelevant).
Because the locations of these fixed points in the action-space depend on the renormalization scheme, each scheme corresponds to a different renormalized trajectory.
The RG flow produced by the blocking steps moves the system towards and along the renormalized trajectory, from the perturbative FP to the infrared fixed point.
At stronger couplings, where we would na\"ively expect backward flow, there might be no ultraviolet FP to drive the RG flow along a renormalized trajectory.
Except in the immediate vicinity of the IRFP, every method that attempts to determine the strong-coupling flow of the gauge coupling (including MCRG two-lattice matching) might then become meaningless.

We determine the bare step-scaling function $s_b(\be_1)$ by matching the lattice actions $S(\be_1, n_b)$ and $S(\be_2, n_{b - 1})$ for systems with bare couplings $\{\be_1, \be_2\}$ after $\{n_b, n_{b - 1}\}$ blocking steps: $s_b(\be_1) \equiv \lim_{n_b \to \infty} \be_1 - \be_2$~\cite{Petropoulos:2012mg}.
When the lattice actions are identical, all observables are identical.
We use the plaquette, the three six-link loops and a planar eight-link loop to perform this matching.
Using short-distance gauge observables allows us to carry out more blocking steps, down to small $2^4$ or $3^4$ lattices.
We minimize finite-volume effects by comparing observables measured on the same blocked volume~\cite{Hasenfratz:2011xn}.
We perform the matching independently for each observable, fitting the data as a cubic function of \be to smoothly interpolate between investigated values of the gauge coupling.

Our finite lattices only allow a few blocking steps, so we must optimize the procedure to reach the renormalized trajectory in as few steps as possible.
In practice, we optimize by tuning some parameter so that consecutive RG blocking steps yield the same $\be_1 - \be_2$, which we identify as $s_b(\be_1)$.
Traditional optimization tunes the RG blocking transformation at each coupling separately, resulting in a different renormalization scheme at each bare coupling $\be_1$: the $s_b$ we obtain is a composite of many different discrete \be functions.
The Wilson flow provides a parameter that we can tune without changing the scheme.

% 5.3
% 4 Flavor Test

\begin{figure}[btp]
  \centering
  \includegraphics[width=0.75\linewidth]{\cnum{5}/fig/dg_nf4_c025_22}
  \caption{\label{fig:Nf4} Continuum extrapolations of the discrete $\be_{\rm lat}$ function of the $N_f = 4$ system at $\gtc(L) = 2.2$ with several different values of the $t$-shift coefficient $\tau_0$.  The dotted lines are independent linear fits at each $\tau_0$, which predict a consistent continuum value.}
\end{figure}
\begin{figure}[btp]
  \centering
  \includegraphics[width=0.75\linewidth]{\cnum{5}/fig/dg_nf4_c025_many}
  \caption{\label{fig:Nf4_many} Continuum extrapolations of the discrete $\be_{\rm lat}$ function of the $N_f = 4$ system for several different $\gtc(L)$ values.  For $\gtc(L) = 1.8$, 2.2 and 2.6 $\tau_0 = -0.02$ is near-optimal, while the larger couplings $\gtc(L) = 3.0$ and 3.4 require $\tau_0 = -0.01$ to remove most $\cO(a^2)$ effects.  The colored points at $(a / L)^2 = 0$ are the continuum extrapolated results, while the black crosses at $(a / L)^2 = 0$ show the corresponding two-loop perturbative predictions.}
\end{figure}

We illustrate the $t$-shift improvement with the $N_f = 4$ SU(3) system.
This theory was recently studied by refs.~\cite{Fodor:2012td, Fodor:2012qh} using gradient flow step scaling with staggered fermions.
The 4-flavor SF running coupling was previously considered in \refcite{Tekin:2010mm} using $\cO(a)$-improved Wilson fermions, and in \refcite{PerezRubio:2010ke} using staggered fermions.
In our calculations we use nHYP-smeared~\cite{Hasenfratz:2001hp, Hasenfratz:2007rf} staggered fermions and a gauge action that includes an adjoint plaquette term in order to move farther away from a well-known spurious fixed point in the adjoint--fundamental plaquette plane~\cite{Cheng:2011ic}.
As in \refcite{Fodor:2012td} we impose anti-periodic BCs in all four directions, which allows us to carry out computations with exactly vanishing fermion mass, $m = 0$.
For the discrete \be function we consider the scale change $s = 3 / 2$ and compare lattice volumes $12^4 \to 18^4$, $16^4 \to 24^4$ and $20^4 \to 30^4$.
We accumulated 500--600 measurements of the gradient flow coupling, with each measurement separated by 10 molecular dynamics time units (MDTU), at 7--8 values of the bare gauge coupling on each volume.
We consider the $c = 0.25$ scheme, as opposed to $c = 0.3$ used in \refcite{Fodor:2012td}, because smaller $c$ gives better statistics at the expense of larger lattice artifacts.
As discussed above, we aim to reduce these lattice artifacts through the non-perturbative improvement we have introduced.
We follow the fitting procedure described in \refcite{Tekin:2010mm}.

Full details of this study will be presented in \refcite{Cheng:2014}.
Here we provide a representative illustration of the $t$-shift optimization.
Figure~\ref{fig:Nf4} shows the dependence of the discrete \be function on $(a / L)^2$ when $\gtc(L) = 2.2$ with several values of the $t$-shift parameter $\tau_0$.
The red triangles correspond to no improvement, $\tau_0 = 0$.
The data are consistent with linear dependence on $a^2$ and extrapolate to 0.262(17), about $2\sigma$ below the two-loop perturbative value of 0.301.
The slope of the extrapolation is already rather small, $b = 11(3)$.
By adding a small shift this slope can be increased or decreased.
With $\tau_0 = -0.02$ no $\cO(a^2)$ effects can be observed -- the corresponding slope is $b = 1.5(3.1)$ -- and we identify this value as near the optimal $\topt$.
The data at different $\tau_0$ extrapolate to the same continuum value, even when the slope $b$ is larger than that for $\tau_0 = 0$.
This is consistent with the expectation that the $t$-shift changes the $\cO(a^2)$ behavior of the system but does not affect the continuum limit.
Since our action produces relatively small $\cO(a^2)$ corrections even without improvement, the $t$-shift optimization has little effect on the continuum extrapolation, though the consistency between different values of $\tau_0$ is reassuring.

It is interesting that the cut-off effects in our unimproved results, characterized by the slope $b$ of the red triangles in \fig{fig:Nf4}, are more than three times smaller than those shown in fig.~4 of \refcite{Fodor:2012td}.
This difference grows to about a factor of four when we consider the larger $c = 0.3$ used in that study, suggesting that the $t$-shift optimization could have a more pronounced effect with the action used in \refcite{Fodor:2012td}.
The cause of the reduced lattice artifacts with our action is not obvious.
Both our action and that used by \refcite{Fodor:2012td} are based on smeared staggered fermions, though we use different smearing schemes.
The different smearing might have an effect, as might the inclusion of the adjoint plaquette term in our gauge action.
This question is worth investigating in the future.

In principle \topt could be different at different \gc couplings but in practice we found little variation.
Figure~\ref{fig:Nf4_many} shows near-optimal continuum extrapolations of the discrete \be function at several values of $\gtc(L)$.
At each $\gtc(L)$ the continuum extrapolated result is consistent within $\sim$$2\sigma$ with the two-loop perturbative prediction, denoted by a black cross in \fig{fig:Nf4_many}.
Comparable consistency with perturbation theory was found in previous studies~\cite{Tekin:2010mm, PerezRubio:2010ke, Fodor:2012td, Fodor:2012qh}.

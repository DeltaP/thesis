% 2.4
% Composite Higgs

One solution to the heirarchy problem is for the Higgs to be a composite composed of particles from a new strongly interacting sector.
This new sector is responsible for electroweak symmetry breaking.
There are many types of theories that use such a modus operandi.
While these theories have the benefits of ... they typically favor a heavier Higgs mass than what has been observed.
Additionally, these theories are highly constrained by precision electroweak measurements.
In the next three subsections I will give a flavor for two classes of composite Higgs theories, introduce the conformal window, and discuss the state of Lattice endeavors in this area.

\subsection{Technicolor and Extended Technicolor}


\subsection{Conformal Window}
\label{sec:sec:conformalwindow}
A general class of strongly interracting theories that are of interest for technicolor and extended technicolor dynamics are Yang-Mills gauge theories.
Pure gauge Yang-Mills theories have $SU(N_c)$ interractions while more general theories include $N_f$ flavors of fermions in some representation.
At low energies QCD is effectively a $SU(3)$ gauge theory with $N_f$=2.
This description includes only the up and down quarks which are nearly massless and respect an approximate $SU(2)$ isospin symmetry.
The most general Yang-Mills Lagrangian with a $SU(N_c)$ local gauge symmetry and $N_f$ flavors of massless fermions in a representation R is:
\begin{equation}
  \Lagr_{YM}=-\frac{1}{4g^2}\sum\limits_{a=1}^{N_c} F^a_{\mu\nu}F^{\mu\nu,a}+\sum\limits_{i=1}^{N_f} \bar\Psi_i\left(i\slashed{D}\right)\Psi_i.  
\end{equation}
$F^a_{\mu\nu}$ is the field strength tensor, shown here for an arbitrary group
\begin{equation}
  F^a_{\mu\nu}=\partial_\mu A^a_\nu-\partial_\nu A^a_\mu + gf^{ijk}A^i_\mu A^j_\nu,
\end{equation}
where $A^a_\mu$ is the field, a is a group index and the structure coefficient $f^{ijk}$ is determined by the commutation relationship
\begin{equation}
  [T^i,T^j]=if^{ijk}T^k,
\end{equation}
and $T^a$ are the generators of the group.
The Lagrangians may look similar however their behavior in both UV and IR can be dramatically different from theory to theory.
While the Lagrangain is classically scale invariant; the quantum theory is not.
This is understood through the beta function,
\begin{equation}
  \beta (g^2)=-\mu\frac{\partial g^2}{\partial \mu}.
\end{equation}
The beta function describes how the gauge coupling evolves as the renormalization scale $\mu$ is changed.
This function can be expanded in perturbation theory and is universal to two loops:
\begin{equation}
  \beta\left(g^2\right)=-\frac{b_1}{16\pi^2}g^4-\frac{b_2}{(16\pi^2)^2}g^6
\end{equation}
Any terms beyond two loops are renormalization scheme dependant and are not relevant for the following discussion.
The values for $b_1$ and $b_2$ are:
\begin{align}
  b_1&=\frac{11}{3}N_c-\frac{4}{3}N_fT(R)\\
  b_2&=\frac{34}{3}N_c^2-\frac{4}{3}T(R)N_f\Big[5N_c+3C_2(R)\Big].
\end{align}
$T(R)$ and $C_2(R)$ are the first and second Casimir invariants and depend on the represenation R of the group.
Table \ref{table:casimir} give these invarients for a few common representations.

\begin{table}
  \centering
  \begin{tabular}{c|ccc}
    Representation    & dim($R$)            & $T(R)$          & $C_2(R)$                \\
    \hline\\
    $F$               & $N$                 & $\frac{1}{2}$   & $\frac{N^2-1}{2N}$      \\\\
    $S_2$             & $\frac{N(N+1)}{2}$  & $\frac{N+2}{2}$ & $\frac{(N+2)(N-1)}{N}$  \\\\
    $A_2$             & $\frac{N(N-1)}{2}$  & $\frac{N-2}{2}$ & $\frac{(N-2)(N+1)}{N}$  \\\\
    $G$               & $N^2-1$             & $N$             & $N$                     \\\\
  \end{tabular}
  \label{table:casimir}
  \caption{this is the caption}
\end{table}

Clearly tuning $N_c$, $N_f$ and $R$ offers a great deal of freedom in specifying the gauge theory.
\begin{figure}[h]
  \centering
  \includegraphics[height=3in]{\cnum{2}/fig/temp.png}
  \caption{}
  \label{fig:confining}
\end{figure}

We can see from equation \ref{eqn:} that if the coefficients $b_1$ and $b_2$ are both positive than the beta function is negative.
In this scenario the theory asymptotically free, confining, and spontaneously breaks chiral symmetry.
The dynamics in the IR will be strongly coupled and nonperturbative.
Figure \ref{fig:confining} shows what this scenario looks like.
This occurs in QCD ($SU(3)$ $N_f=2\mbox{ or }3$) is an example of a theory where the beta function is negative.

\begin{figure}[h]
  \centering
  \includegraphics[height=3in]{\cnum{2}/fig/temp.png}
  \caption{}
  \label{fig:conformal}
\end{figure}
If we keep $b_1$ positive and allow $b_2$ to become negative we can force the two terms to compete.
Such a beta function would start out negative, then at some coupling it would pass through a local minimum after which it would grow.
Eventually the beta function would have a zero where $\beta=0$.
Perturbatively this zero occurs at $g^2_*=-b_1/b2$ and is a Banks-Zaks infrared fixed point.
This is illustrated in \ref{fig:conformal}, such a theory is governed by the conformal dynamcis at the infrared fixed point and is scale invarient.
Unlike confining theories, such theories do not support bound states of particles.

\begin{figure}[h]
  \centering
  \includegraphics[height=3in]{\cnum{2}/fig/temp.png}
  \caption{}
  \label{fig:trivial}
\end{figure}
Allowing both $b_1$ and $b_2$ to be negative results in a trivial theory.
Such a theory, shown in \ref{fig:trivial} is not asymptotically free.  
Such a beta function is very similar to that of QED.
Perturbatively this occurs when $N_f>$.

The region in theory space between where the IRFP appears and where asymptotic freedom is lost is reffered to as the conformal window.
Limmiting our consideration to theories using the fundamental representaiton fermions and $N_c=3$ we see that the conformal window is between $N_f= \mbox{ and } N_f=$.
It is important to note that because of the nature of these theories perturbation theory is not reliable and these numbers only serve as motivation for where to look.

\begin{figure}[h]
  \centering
  \includegraphics[height=3in]{\cnum{2}/fig/temp.png}
  \caption{}
  \label{fig:walking}
\end{figure}
The conformal window itself is not a very interesting system to study but understanding the lower bound of the conformal window is of great interet for potential technicolor theories.
Right below the conformal window it is believed that a walking theory can exist.
In such a theory, shown in figure \ref{fig:walking} the beta function would start out like the conformal scenario.
However, as the theory approaches the IRFP chiral symmetry is spontateously broken and the beta function would turn around away from the IRFP.
This scenario allows 

\subsection{Lattice Studies}
There has been a growing interest in studying beyond the standard model theories on the lattices over the past decade.
To date there are a number of active lattice groups carrying out caluclations in this field.
It is healthy that such a variety of theories have been studied usinging different methods.
Beyond standard model physics is inherently a difficult subject to study because we don't know the answer before hand.
Taking a holisitic approach and finding consensus is vital to our understanding.
Additionally, as more of these theories have been studied, vital improvements have been made.
This is true of the lattice code base which prior to BSM studies was highly optimized and sometimes only available for QCD simulations.
Improvements hav also been made in the way the data is analyzed.
I will discuss two improvements we have developed in chapters \ref{ch:}-\ref{ch:}

As I mentioned above there are a number of ways to appraoch BSM lattice studies.
One class of method which I have been involved with is to calculate the beta function directly.
There are several ways to calculate the beta function on the lattice.
One technique is called the Schr\"{o}dinger functional.
The Schr\"{o}dinger functional has been widly used to study several theories.
Another method is 



Another class of methods involve lattice spectroscopy.


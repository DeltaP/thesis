% 2.2
% Higgs Mechanism

In the standard model the $W^{\pm}$ and $Z$ bosons are massive but the photon is massless.
Since gauge invariance dictates that a mass term is forbidden in the Lagrangian this seems to pose a problem.
Disaster is averted because $SU(2)_WxU(1)_Y$ is spontaneously broken to $U(1)_{EM}$.
In the standard model this is accomplished by adding a complex scalar doublet field to the theory.
The field is named the Higgs field after one of its discoverers.
The following discussion shows how electroweak symmetry breaking is facilitated by the Higgs \cite{ds 52, 53, peskin}.

Lets introduce a complex elementary scalar doublet $\Phi=\left(\begin{matrix}\phi^+\\\phi_0\end{matrix}\right)$ that transforms in the (1,2,$\frac{1}{2}$) representation of $SU(3)_c\times SU(2)_L\times U(1)_Y$.
We allow all terms in the Lagrangian that have mass dimension $\leq 4$, thus ignoring irrelevant operators,
\begin{equation}
  \Lagr_H=(D_\mu\Phi)^\dagger(D^\mu\Phi)+\mu^2\Phi^\dagger\Phi-|\lambda|(\Phi^\dagger\Phi)^2,
\end{equation}
with covariant derivative
\begin{equation}
  D_\mu\Phi=(\partial_\mu-igA^a_\mu\tau^a-\frac{ig'}{2}B_\mu)\Phi.
\end{equation}
$A^a_\mu$ and $B_\mu$ are the $SU(2)$ and $U(1)$ gauge bosons respectively.
The coupling constant $g$ belongs to $SU(2)$ and the coupling constant $g'$ belongs to $U(1)$.
Finally $\tau^a$ are the generators of $SU(2)$ and are related to the Pauli matrices
\begin{equation}
  \tau^a=\frac{1}{2}\sigma^a.
\end{equation}

We can now identify the potential $V(\Phi)$ as
\begin{equation}
  V(\Phi)=-\mu^2\Phi^\dagger\Phi+|\lambda|(\Phi^\dagger\Phi)^2,
\end{equation}
this potential is shown in figure \ref{fig:higgspot}.
As long as $\mu^2$ is positive the potential has a spontaneously broken symmetry.
Gauge invariance allows us to choose the vacuum state to correspond to the vacuum expectation value
\begin{equation}
  \left<\Phi\right>_0=\frac{\nu}{\sqrt{2}}\left(\begin{matrix}0 \\ \nu\end{matrix}\right),
\end{equation}
where $\nu=\sqrt{\frac{\mu^2}{|\lambda|}}$.
Moreover, $\nu$ is related to the Fermi constant, $G_F$, by
\begin{equation}
  \nu=\frac{1}{\sqrt{\sqrt{2}G_F}}\approx 246 \mbox{ GeV}.
\end{equation}

A complete gauge transformation in this theory is
\begin{equation}
  \Phi\rightarrow e^{i\alpha^a(x)\tau^a}e^{i\beta(x)/2}\Phi.
\end{equation}
Through a clever choice of $\alpha^1=\alpha^2=0$ and $\alpha^3=\beta$ we see that $\left<\Phi\right>$ is invariant.
Therefore the theory still contains an unbroken U(1) symmetry which we identify with electromagnetism.





The Higgs Boson, a particle excitations of the Higgs field, was recently discovered at the Large Hadron Collider (LHC).
The Higgs Boson was recently discovered at the Large Hadron Collider.
Current experimental results pin the Higgs mass at


